
% Default to the notebook output style

    


% Inherit from the specified cell style.




    
\documentclass[11pt]{article}

    
    
    \usepackage[T1]{fontenc}
    % Nicer default font (+ math font) than Computer Modern for most use cases
    \usepackage{mathpazo}

    % Basic figure setup, for now with no caption control since it's done
    % automatically by Pandoc (which extracts ![](path) syntax from Markdown).
    \usepackage{graphicx}
    % We will generate all images so they have a width \maxwidth. This means
    % that they will get their normal width if they fit onto the page, but
    % are scaled down if they would overflow the margins.
    \makeatletter
    \def\maxwidth{\ifdim\Gin@nat@width>\linewidth\linewidth
    \else\Gin@nat@width\fi}
    \makeatother
    \let\Oldincludegraphics\includegraphics
    % Set max figure width to be 80% of text width, for now hardcoded.
    \renewcommand{\includegraphics}[1]{\Oldincludegraphics[width=.8\maxwidth]{#1}}
    % Ensure that by default, figures have no caption (until we provide a
    % proper Figure object with a Caption API and a way to capture that
    % in the conversion process - todo).
    \usepackage{caption}
    \DeclareCaptionLabelFormat{nolabel}{}
    \captionsetup{labelformat=nolabel}

    \usepackage{adjustbox} % Used to constrain images to a maximum size 
    \usepackage{xcolor} % Allow colors to be defined
    \usepackage{enumerate} % Needed for markdown enumerations to work
    \usepackage{geometry} % Used to adjust the document margins
    \usepackage{amsmath} % Equations
    \usepackage{amssymb} % Equations
    \usepackage{textcomp} % defines textquotesingle
    % Hack from http://tex.stackexchange.com/a/47451/13684:
    \AtBeginDocument{%
        \def\PYZsq{\textquotesingle}% Upright quotes in Pygmentized code
    }
    \usepackage{upquote} % Upright quotes for verbatim code
    \usepackage{eurosym} % defines \euro
    \usepackage[mathletters]{ucs} % Extended unicode (utf-8) support
    \usepackage[utf8x]{inputenc} % Allow utf-8 characters in the tex document
    \usepackage{fancyvrb} % verbatim replacement that allows latex
    \usepackage{grffile} % extends the file name processing of package graphics 
                         % to support a larger range 
    % The hyperref package gives us a pdf with properly built
    % internal navigation ('pdf bookmarks' for the table of contents,
    % internal cross-reference links, web links for URLs, etc.)
    \usepackage{hyperref}
    \usepackage{longtable} % longtable support required by pandoc >1.10
    \usepackage{booktabs}  % table support for pandoc > 1.12.2
    \usepackage[inline]{enumitem} % IRkernel/repr support (it uses the enumerate* environment)
    \usepackage[normalem]{ulem} % ulem is needed to support strikethroughs (\sout)
                                % normalem makes italics be italics, not underlines
    

    
    
    % Colors for the hyperref package
    \definecolor{urlcolor}{rgb}{0,.145,.698}
    \definecolor{linkcolor}{rgb}{.71,0.21,0.01}
    \definecolor{citecolor}{rgb}{.12,.54,.11}

    % ANSI colors
    \definecolor{ansi-black}{HTML}{3E424D}
    \definecolor{ansi-black-intense}{HTML}{282C36}
    \definecolor{ansi-red}{HTML}{E75C58}
    \definecolor{ansi-red-intense}{HTML}{B22B31}
    \definecolor{ansi-green}{HTML}{00A250}
    \definecolor{ansi-green-intense}{HTML}{007427}
    \definecolor{ansi-yellow}{HTML}{DDB62B}
    \definecolor{ansi-yellow-intense}{HTML}{B27D12}
    \definecolor{ansi-blue}{HTML}{208FFB}
    \definecolor{ansi-blue-intense}{HTML}{0065CA}
    \definecolor{ansi-magenta}{HTML}{D160C4}
    \definecolor{ansi-magenta-intense}{HTML}{A03196}
    \definecolor{ansi-cyan}{HTML}{60C6C8}
    \definecolor{ansi-cyan-intense}{HTML}{258F8F}
    \definecolor{ansi-white}{HTML}{C5C1B4}
    \definecolor{ansi-white-intense}{HTML}{A1A6B2}

    % commands and environments needed by pandoc snippets
    % extracted from the output of `pandoc -s`
    \providecommand{\tightlist}{%
      \setlength{\itemsep}{0pt}\setlength{\parskip}{0pt}}
    \DefineVerbatimEnvironment{Highlighting}{Verbatim}{commandchars=\\\{\}}
    % Add ',fontsize=\small' for more characters per line
    \newenvironment{Shaded}{}{}
    \newcommand{\KeywordTok}[1]{\textcolor[rgb]{0.00,0.44,0.13}{\textbf{{#1}}}}
    \newcommand{\DataTypeTok}[1]{\textcolor[rgb]{0.56,0.13,0.00}{{#1}}}
    \newcommand{\DecValTok}[1]{\textcolor[rgb]{0.25,0.63,0.44}{{#1}}}
    \newcommand{\BaseNTok}[1]{\textcolor[rgb]{0.25,0.63,0.44}{{#1}}}
    \newcommand{\FloatTok}[1]{\textcolor[rgb]{0.25,0.63,0.44}{{#1}}}
    \newcommand{\CharTok}[1]{\textcolor[rgb]{0.25,0.44,0.63}{{#1}}}
    \newcommand{\StringTok}[1]{\textcolor[rgb]{0.25,0.44,0.63}{{#1}}}
    \newcommand{\CommentTok}[1]{\textcolor[rgb]{0.38,0.63,0.69}{\textit{{#1}}}}
    \newcommand{\OtherTok}[1]{\textcolor[rgb]{0.00,0.44,0.13}{{#1}}}
    \newcommand{\AlertTok}[1]{\textcolor[rgb]{1.00,0.00,0.00}{\textbf{{#1}}}}
    \newcommand{\FunctionTok}[1]{\textcolor[rgb]{0.02,0.16,0.49}{{#1}}}
    \newcommand{\RegionMarkerTok}[1]{{#1}}
    \newcommand{\ErrorTok}[1]{\textcolor[rgb]{1.00,0.00,0.00}{\textbf{{#1}}}}
    \newcommand{\NormalTok}[1]{{#1}}
    
    % Additional commands for more recent versions of Pandoc
    \newcommand{\ConstantTok}[1]{\textcolor[rgb]{0.53,0.00,0.00}{{#1}}}
    \newcommand{\SpecialCharTok}[1]{\textcolor[rgb]{0.25,0.44,0.63}{{#1}}}
    \newcommand{\VerbatimStringTok}[1]{\textcolor[rgb]{0.25,0.44,0.63}{{#1}}}
    \newcommand{\SpecialStringTok}[1]{\textcolor[rgb]{0.73,0.40,0.53}{{#1}}}
    \newcommand{\ImportTok}[1]{{#1}}
    \newcommand{\DocumentationTok}[1]{\textcolor[rgb]{0.73,0.13,0.13}{\textit{{#1}}}}
    \newcommand{\AnnotationTok}[1]{\textcolor[rgb]{0.38,0.63,0.69}{\textbf{\textit{{#1}}}}}
    \newcommand{\CommentVarTok}[1]{\textcolor[rgb]{0.38,0.63,0.69}{\textbf{\textit{{#1}}}}}
    \newcommand{\VariableTok}[1]{\textcolor[rgb]{0.10,0.09,0.49}{{#1}}}
    \newcommand{\ControlFlowTok}[1]{\textcolor[rgb]{0.00,0.44,0.13}{\textbf{{#1}}}}
    \newcommand{\OperatorTok}[1]{\textcolor[rgb]{0.40,0.40,0.40}{{#1}}}
    \newcommand{\BuiltInTok}[1]{{#1}}
    \newcommand{\ExtensionTok}[1]{{#1}}
    \newcommand{\PreprocessorTok}[1]{\textcolor[rgb]{0.74,0.48,0.00}{{#1}}}
    \newcommand{\AttributeTok}[1]{\textcolor[rgb]{0.49,0.56,0.16}{{#1}}}
    \newcommand{\InformationTok}[1]{\textcolor[rgb]{0.38,0.63,0.69}{\textbf{\textit{{#1}}}}}
    \newcommand{\WarningTok}[1]{\textcolor[rgb]{0.38,0.63,0.69}{\textbf{\textit{{#1}}}}}
    
    
    % Define a nice break command that doesn't care if a line doesn't already
    % exist.
    \def\br{\hspace*{\fill} \\* }
    % Math Jax compatability definitions
    \def\gt{>}
    \def\lt{<}
    % Document parameters
    \title{Analysis of Spike-Field Coherence}
    
    
    

    % Pygments definitions
    
\makeatletter
\def\PY@reset{\let\PY@it=\relax \let\PY@bf=\relax%
    \let\PY@ul=\relax \let\PY@tc=\relax%
    \let\PY@bc=\relax \let\PY@ff=\relax}
\def\PY@tok#1{\csname PY@tok@#1\endcsname}
\def\PY@toks#1+{\ifx\relax#1\empty\else%
    \PY@tok{#1}\expandafter\PY@toks\fi}
\def\PY@do#1{\PY@bc{\PY@tc{\PY@ul{%
    \PY@it{\PY@bf{\PY@ff{#1}}}}}}}
\def\PY#1#2{\PY@reset\PY@toks#1+\relax+\PY@do{#2}}

\expandafter\def\csname PY@tok@w\endcsname{\def\PY@tc##1{\textcolor[rgb]{0.73,0.73,0.73}{##1}}}
\expandafter\def\csname PY@tok@c\endcsname{\let\PY@it=\textit\def\PY@tc##1{\textcolor[rgb]{0.25,0.50,0.50}{##1}}}
\expandafter\def\csname PY@tok@cp\endcsname{\def\PY@tc##1{\textcolor[rgb]{0.74,0.48,0.00}{##1}}}
\expandafter\def\csname PY@tok@k\endcsname{\let\PY@bf=\textbf\def\PY@tc##1{\textcolor[rgb]{0.00,0.50,0.00}{##1}}}
\expandafter\def\csname PY@tok@kp\endcsname{\def\PY@tc##1{\textcolor[rgb]{0.00,0.50,0.00}{##1}}}
\expandafter\def\csname PY@tok@kt\endcsname{\def\PY@tc##1{\textcolor[rgb]{0.69,0.00,0.25}{##1}}}
\expandafter\def\csname PY@tok@o\endcsname{\def\PY@tc##1{\textcolor[rgb]{0.40,0.40,0.40}{##1}}}
\expandafter\def\csname PY@tok@ow\endcsname{\let\PY@bf=\textbf\def\PY@tc##1{\textcolor[rgb]{0.67,0.13,1.00}{##1}}}
\expandafter\def\csname PY@tok@nb\endcsname{\def\PY@tc##1{\textcolor[rgb]{0.00,0.50,0.00}{##1}}}
\expandafter\def\csname PY@tok@nf\endcsname{\def\PY@tc##1{\textcolor[rgb]{0.00,0.00,1.00}{##1}}}
\expandafter\def\csname PY@tok@nc\endcsname{\let\PY@bf=\textbf\def\PY@tc##1{\textcolor[rgb]{0.00,0.00,1.00}{##1}}}
\expandafter\def\csname PY@tok@nn\endcsname{\let\PY@bf=\textbf\def\PY@tc##1{\textcolor[rgb]{0.00,0.00,1.00}{##1}}}
\expandafter\def\csname PY@tok@ne\endcsname{\let\PY@bf=\textbf\def\PY@tc##1{\textcolor[rgb]{0.82,0.25,0.23}{##1}}}
\expandafter\def\csname PY@tok@nv\endcsname{\def\PY@tc##1{\textcolor[rgb]{0.10,0.09,0.49}{##1}}}
\expandafter\def\csname PY@tok@no\endcsname{\def\PY@tc##1{\textcolor[rgb]{0.53,0.00,0.00}{##1}}}
\expandafter\def\csname PY@tok@nl\endcsname{\def\PY@tc##1{\textcolor[rgb]{0.63,0.63,0.00}{##1}}}
\expandafter\def\csname PY@tok@ni\endcsname{\let\PY@bf=\textbf\def\PY@tc##1{\textcolor[rgb]{0.60,0.60,0.60}{##1}}}
\expandafter\def\csname PY@tok@na\endcsname{\def\PY@tc##1{\textcolor[rgb]{0.49,0.56,0.16}{##1}}}
\expandafter\def\csname PY@tok@nt\endcsname{\let\PY@bf=\textbf\def\PY@tc##1{\textcolor[rgb]{0.00,0.50,0.00}{##1}}}
\expandafter\def\csname PY@tok@nd\endcsname{\def\PY@tc##1{\textcolor[rgb]{0.67,0.13,1.00}{##1}}}
\expandafter\def\csname PY@tok@s\endcsname{\def\PY@tc##1{\textcolor[rgb]{0.73,0.13,0.13}{##1}}}
\expandafter\def\csname PY@tok@sd\endcsname{\let\PY@it=\textit\def\PY@tc##1{\textcolor[rgb]{0.73,0.13,0.13}{##1}}}
\expandafter\def\csname PY@tok@si\endcsname{\let\PY@bf=\textbf\def\PY@tc##1{\textcolor[rgb]{0.73,0.40,0.53}{##1}}}
\expandafter\def\csname PY@tok@se\endcsname{\let\PY@bf=\textbf\def\PY@tc##1{\textcolor[rgb]{0.73,0.40,0.13}{##1}}}
\expandafter\def\csname PY@tok@sr\endcsname{\def\PY@tc##1{\textcolor[rgb]{0.73,0.40,0.53}{##1}}}
\expandafter\def\csname PY@tok@ss\endcsname{\def\PY@tc##1{\textcolor[rgb]{0.10,0.09,0.49}{##1}}}
\expandafter\def\csname PY@tok@sx\endcsname{\def\PY@tc##1{\textcolor[rgb]{0.00,0.50,0.00}{##1}}}
\expandafter\def\csname PY@tok@m\endcsname{\def\PY@tc##1{\textcolor[rgb]{0.40,0.40,0.40}{##1}}}
\expandafter\def\csname PY@tok@gh\endcsname{\let\PY@bf=\textbf\def\PY@tc##1{\textcolor[rgb]{0.00,0.00,0.50}{##1}}}
\expandafter\def\csname PY@tok@gu\endcsname{\let\PY@bf=\textbf\def\PY@tc##1{\textcolor[rgb]{0.50,0.00,0.50}{##1}}}
\expandafter\def\csname PY@tok@gd\endcsname{\def\PY@tc##1{\textcolor[rgb]{0.63,0.00,0.00}{##1}}}
\expandafter\def\csname PY@tok@gi\endcsname{\def\PY@tc##1{\textcolor[rgb]{0.00,0.63,0.00}{##1}}}
\expandafter\def\csname PY@tok@gr\endcsname{\def\PY@tc##1{\textcolor[rgb]{1.00,0.00,0.00}{##1}}}
\expandafter\def\csname PY@tok@ge\endcsname{\let\PY@it=\textit}
\expandafter\def\csname PY@tok@gs\endcsname{\let\PY@bf=\textbf}
\expandafter\def\csname PY@tok@gp\endcsname{\let\PY@bf=\textbf\def\PY@tc##1{\textcolor[rgb]{0.00,0.00,0.50}{##1}}}
\expandafter\def\csname PY@tok@go\endcsname{\def\PY@tc##1{\textcolor[rgb]{0.53,0.53,0.53}{##1}}}
\expandafter\def\csname PY@tok@gt\endcsname{\def\PY@tc##1{\textcolor[rgb]{0.00,0.27,0.87}{##1}}}
\expandafter\def\csname PY@tok@err\endcsname{\def\PY@bc##1{\setlength{\fboxsep}{0pt}\fcolorbox[rgb]{1.00,0.00,0.00}{1,1,1}{\strut ##1}}}
\expandafter\def\csname PY@tok@kc\endcsname{\let\PY@bf=\textbf\def\PY@tc##1{\textcolor[rgb]{0.00,0.50,0.00}{##1}}}
\expandafter\def\csname PY@tok@kd\endcsname{\let\PY@bf=\textbf\def\PY@tc##1{\textcolor[rgb]{0.00,0.50,0.00}{##1}}}
\expandafter\def\csname PY@tok@kn\endcsname{\let\PY@bf=\textbf\def\PY@tc##1{\textcolor[rgb]{0.00,0.50,0.00}{##1}}}
\expandafter\def\csname PY@tok@kr\endcsname{\let\PY@bf=\textbf\def\PY@tc##1{\textcolor[rgb]{0.00,0.50,0.00}{##1}}}
\expandafter\def\csname PY@tok@bp\endcsname{\def\PY@tc##1{\textcolor[rgb]{0.00,0.50,0.00}{##1}}}
\expandafter\def\csname PY@tok@fm\endcsname{\def\PY@tc##1{\textcolor[rgb]{0.00,0.00,1.00}{##1}}}
\expandafter\def\csname PY@tok@vc\endcsname{\def\PY@tc##1{\textcolor[rgb]{0.10,0.09,0.49}{##1}}}
\expandafter\def\csname PY@tok@vg\endcsname{\def\PY@tc##1{\textcolor[rgb]{0.10,0.09,0.49}{##1}}}
\expandafter\def\csname PY@tok@vi\endcsname{\def\PY@tc##1{\textcolor[rgb]{0.10,0.09,0.49}{##1}}}
\expandafter\def\csname PY@tok@vm\endcsname{\def\PY@tc##1{\textcolor[rgb]{0.10,0.09,0.49}{##1}}}
\expandafter\def\csname PY@tok@sa\endcsname{\def\PY@tc##1{\textcolor[rgb]{0.73,0.13,0.13}{##1}}}
\expandafter\def\csname PY@tok@sb\endcsname{\def\PY@tc##1{\textcolor[rgb]{0.73,0.13,0.13}{##1}}}
\expandafter\def\csname PY@tok@sc\endcsname{\def\PY@tc##1{\textcolor[rgb]{0.73,0.13,0.13}{##1}}}
\expandafter\def\csname PY@tok@dl\endcsname{\def\PY@tc##1{\textcolor[rgb]{0.73,0.13,0.13}{##1}}}
\expandafter\def\csname PY@tok@s2\endcsname{\def\PY@tc##1{\textcolor[rgb]{0.73,0.13,0.13}{##1}}}
\expandafter\def\csname PY@tok@sh\endcsname{\def\PY@tc##1{\textcolor[rgb]{0.73,0.13,0.13}{##1}}}
\expandafter\def\csname PY@tok@s1\endcsname{\def\PY@tc##1{\textcolor[rgb]{0.73,0.13,0.13}{##1}}}
\expandafter\def\csname PY@tok@mb\endcsname{\def\PY@tc##1{\textcolor[rgb]{0.40,0.40,0.40}{##1}}}
\expandafter\def\csname PY@tok@mf\endcsname{\def\PY@tc##1{\textcolor[rgb]{0.40,0.40,0.40}{##1}}}
\expandafter\def\csname PY@tok@mh\endcsname{\def\PY@tc##1{\textcolor[rgb]{0.40,0.40,0.40}{##1}}}
\expandafter\def\csname PY@tok@mi\endcsname{\def\PY@tc##1{\textcolor[rgb]{0.40,0.40,0.40}{##1}}}
\expandafter\def\csname PY@tok@il\endcsname{\def\PY@tc##1{\textcolor[rgb]{0.40,0.40,0.40}{##1}}}
\expandafter\def\csname PY@tok@mo\endcsname{\def\PY@tc##1{\textcolor[rgb]{0.40,0.40,0.40}{##1}}}
\expandafter\def\csname PY@tok@ch\endcsname{\let\PY@it=\textit\def\PY@tc##1{\textcolor[rgb]{0.25,0.50,0.50}{##1}}}
\expandafter\def\csname PY@tok@cm\endcsname{\let\PY@it=\textit\def\PY@tc##1{\textcolor[rgb]{0.25,0.50,0.50}{##1}}}
\expandafter\def\csname PY@tok@cpf\endcsname{\let\PY@it=\textit\def\PY@tc##1{\textcolor[rgb]{0.25,0.50,0.50}{##1}}}
\expandafter\def\csname PY@tok@c1\endcsname{\let\PY@it=\textit\def\PY@tc##1{\textcolor[rgb]{0.25,0.50,0.50}{##1}}}
\expandafter\def\csname PY@tok@cs\endcsname{\let\PY@it=\textit\def\PY@tc##1{\textcolor[rgb]{0.25,0.50,0.50}{##1}}}

\def\PYZbs{\char`\\}
\def\PYZus{\char`\_}
\def\PYZob{\char`\{}
\def\PYZcb{\char`\}}
\def\PYZca{\char`\^}
\def\PYZam{\char`\&}
\def\PYZlt{\char`\<}
\def\PYZgt{\char`\>}
\def\PYZsh{\char`\#}
\def\PYZpc{\char`\%}
\def\PYZdl{\char`\$}
\def\PYZhy{\char`\-}
\def\PYZsq{\char`\'}
\def\PYZdq{\char`\"}
\def\PYZti{\char`\~}
% for compatibility with earlier versions
\def\PYZat{@}
\def\PYZlb{[}
\def\PYZrb{]}
\makeatother


    % Exact colors from NB
    \definecolor{incolor}{rgb}{0.0, 0.0, 0.5}
    \definecolor{outcolor}{rgb}{0.545, 0.0, 0.0}



    
    % Prevent overflowing lines due to hard-to-break entities
    \sloppy 
    % Setup hyperref package
    \hypersetup{
      breaklinks=true,  % so long urls are correctly broken across lines
      colorlinks=true,
      urlcolor=urlcolor,
      linkcolor=linkcolor,
      citecolor=citecolor,
      }
    % Slightly bigger margins than the latex defaults
    
    \geometry{verbose,tmargin=1in,bmargin=1in,lmargin=1in,rmargin=1in}
    
    

    \begin{document}
    
    
    \maketitle
    
    

    
    \section{\texorpdfstring{Analysis of rhythmic activity \emph{for the
practicing neuroscientist}
}{Analysis of rhythmic activity for the practicing neuroscientist }}\label{analysis-of-rhythmic-activity-for-the-practicing-neuroscientist}

    \emph{\textbf{Synopsis}} \textbf{Data:} 100 trials of 1 s of local field
potential and spike train data sampled at 1000 Hz.

\textbf{Goal:} Characterize the coupling between the spike and field
activity.

\textbf{Tools:} Fourier transform, spectrum, coherence, phase,
generalized linear models.

    \begin{itemize}
\tightlist
\item
  Section \ref{}
\item
  Section \ref{data-analysis}

  \begin{itemize}
  \tightlist
  \item
    Section \ref{visual-inspection}
  \item
    Section \ref{mean}
  \item
    Section \ref{autocovariance}
  \item
    Section \ref{power-spectral-density}

    \begin{itemize}
    \tightlist
    \item
      Section \ref{spectrum}
    \item
      Section \ref{dft}
    \item
      Section \ref{nyquist-frequency}
    \item
      Section \ref{frequency-resolution}
    \end{itemize}
  \item
    Section \ref{decibel-scaling}
  \item
    Section \ref{the-spectrogram}
  \end{itemize}
\item
  Section \ref{summary}
\end{itemize}

    \subsection{Introduction}\label{introduction}

In the previous sections, we focused on two types of data: field data
(e.g., EEG, ECoG, LFP) and spiking data (i.e., action potentials), and
we developed techniques to analyze these data. In this chapter, we
consider the simultaneous observation of both data types. We analyze
these multiscale data using the techniques developed in previous
chapters and focus specifically on computing the coherence between the
spike and field recordings. Understanding the relations between activity
recorded at different spatial scales (i.e., a macroscopic field and
microscopic spikes) remains an active research area.

\subsubsection{Case study data}\label{case-study-data}

Our experimental collaborator has implanted an electrode in rat
hippocampus as the animal performs a task requiring navigation and
decision making. From these data, he is able to extract the local field
potential (LFP) as well as the spiking activity of a single neuron. He
would like to characterize how these multiscale data---the population
field activity and the single neuron spiking activity---relate. Based on
existing evidence in the literature and experimental intuition, he
expects that rhythmic activity in the LFP impacts the probability that a
spike will occur. As his collaborator, we will help him to develop tools
to examine this hypothesis. He provides us with 100 trials of
simultaneous LFP and spike train data with a sampling frequency of 1000
Hz. The duration of each trial is 1 s, corresponding to a fixed temporal
interval following a particular decision of the rat.

\subsubsection{Goals}\label{goals}

Our goal is to understand the coupling between the spiking activity and
the LFP following the stimulus. To do so, we analyze the multiscale data
recorded simultaneously. To assess this coupling, we will start with two
visualizations of the data: the spike-triggered average and the
field-triggered average. We then compute the spike-field coherence, a
coupling measure that builds upon previous development of the Fourier
transform and spectrum. We also examine how the firing rate impacts
measures of coupling and how to mitigate this impact.

\subsubsection{Tools}\label{tools}

In this chapter, we focus primarily on computing the spike-field
coherence. Development of this measure makes use of skills developed in
previous sections. In computing the spike-field coherence, we continue
to utilize the Fourier transform. We also consider how generalized
linear models (GLMs) can be used to construct a measure of spike-field
association with an important advantage over the spike-field coherence.

    \subsection{Data analysis}\label{data-analysis}

We will go through the following steps to analyze the data:

\begin{enumerate}
\def\labelenumi{\arabic{enumi}.}
\tightlist
\item
  Section \ref{visual-inspection}
\item
  Section \ref{mean}
\item
  Section \ref{autocovariance}
\item
  Section \ref{power-spectral-density}
\item
  Section \ref{decibel-scaling}
\item
  Section \ref{the-spectrogram}
\end{enumerate}

    \subsubsection{Step 1: Visual
inspection}\label{step-1-visual-inspection}

    We begin the analysis by visualizing examples of the simultaneously
recorded spike train and LFP data. Let's load these multi- scale data
into MATLAB and plot the activity of the first trial:

    \begin{Verbatim}[commandchars=\\\{\}]
{\color{incolor}In [{\color{incolor}16}]:} \PY{c+c1}{\PYZsh{} Prepare the modules and plot settings}
         \PY{k+kn}{import} \PY{n+nn}{scipy}\PY{n+nn}{.}\PY{n+nn}{io} \PY{k}{as} \PY{n+nn}{sio}
         \PY{k+kn}{from} \PY{n+nn}{scipy} \PY{k}{import} \PY{n}{signal}
         \PY{k+kn}{import} \PY{n+nn}{numpy} \PY{k}{as} \PY{n+nn}{np}
         \PY{k+kn}{import} \PY{n+nn}{matplotlib}\PY{n+nn}{.}\PY{n+nn}{pyplot} \PY{k}{as} \PY{n+nn}{plt}
         \PY{k+kn}{from} \PY{n+nn}{matplotlib}\PY{n+nn}{.}\PY{n+nn}{pyplot} \PY{k}{import} \PY{n}{xlabel}\PY{p}{,} \PY{n}{ylabel}\PY{p}{,} \PY{n}{plot}\PY{p}{,} \PY{n}{show}\PY{p}{,} \PY{n}{title}
         \PY{k+kn}{from} \PY{n+nn}{matplotlib} \PY{k}{import} \PY{n}{rcParams}
         \PY{k+kn}{import} \PY{n+nn}{setup}
         \PY{o}{\PYZpc{}}\PY{k}{matplotlib} inline
         \PY{n}{rcParams}\PY{p}{[}\PY{l+s+s1}{\PYZsq{}}\PY{l+s+s1}{figure.figsize}\PY{l+s+s1}{\PYZsq{}}\PY{p}{]} \PY{o}{=} \PY{p}{(}\PY{l+m+mi}{12}\PY{p}{,}\PY{l+m+mi}{3}\PY{p}{)}
\end{Verbatim}


    \begin{Verbatim}[commandchars=\\\{\}]
{\color{incolor}In [{\color{incolor}6}]:} \PY{n}{setup}\PY{o}{.}\PY{n}{set\PYZus{}defaults}\PY{p}{(}\PY{p}{)}
\end{Verbatim}


\begin{Verbatim}[commandchars=\\\{\}]
{\color{outcolor}Out[{\color{outcolor}6}]:} <IPython.core.display.HTML object>
\end{Verbatim}
            
    \begin{Verbatim}[commandchars=\\\{\}]
{\color{incolor}In [{\color{incolor}7}]:} \PY{n}{data} \PY{o}{=} \PY{n}{sio}\PY{o}{.}\PY{n}{loadmat}\PY{p}{(}\PY{l+s+s1}{\PYZsq{}}\PY{l+s+s1}{Ch11\PYZhy{}spikes\PYZhy{}LFP\PYZhy{}1.mat}\PY{l+s+s1}{\PYZsq{}}\PY{p}{)}  \PY{c+c1}{\PYZsh{} Load the multiscale data,}
        \PY{n}{y} \PY{o}{=} \PY{n}{data}\PY{p}{[}\PY{l+s+s1}{\PYZsq{}}\PY{l+s+s1}{y}\PY{l+s+s1}{\PYZsq{}}\PY{p}{]}
        \PY{n}{t} \PY{o}{=} \PY{n}{data}\PY{p}{[}\PY{l+s+s1}{\PYZsq{}}\PY{l+s+s1}{t}\PY{l+s+s1}{\PYZsq{}}\PY{p}{]}\PY{o}{.}\PY{n}{reshape}\PY{p}{(}\PY{o}{\PYZhy{}}\PY{l+m+mi}{1}\PY{p}{)}
        \PY{n}{n} \PY{o}{=} \PY{n}{data}\PY{p}{[}\PY{l+s+s1}{\PYZsq{}}\PY{l+s+s1}{n}\PY{l+s+s1}{\PYZsq{}}\PY{p}{]}
        \PY{n}{plot}\PY{p}{(}\PY{n}{t}\PY{p}{,}\PY{n}{y}\PY{p}{[}\PY{l+m+mi}{1}\PY{p}{,}\PY{p}{:}\PY{p}{]}\PY{p}{)}                               \PY{c+c1}{\PYZsh{} ... and visualize it.}
        \PY{n}{plot}\PY{p}{(}\PY{n}{t}\PY{p}{,}\PY{n}{n}\PY{p}{[}\PY{l+m+mi}{1}\PY{p}{,}\PY{p}{:}\PY{p}{]}\PY{p}{)}
        \PY{n}{xlabel}\PY{p}{(}\PY{l+s+s1}{\PYZsq{}}\PY{l+s+s1}{Time [s]}\PY{l+s+s1}{\PYZsq{}}\PY{p}{)}
        \PY{n}{plt}\PY{o}{.}\PY{n}{autoscale}\PY{p}{(}\PY{n}{tight}\PY{o}{=}\PY{k+kc}{True}\PY{p}{)}                    \PY{c+c1}{\PYZsh{} ... with white space minimized.}
\end{Verbatim}


    \begin{center}
    \adjustimage{max size={0.9\linewidth}{0.9\paperheight}}{output_9_0.png}
    \end{center}
    { \hspace*{\fill} \\}
    
    \textbf{Array shapes:} The \texttt{reshape()} function lets us change
the shape of an array. \texttt{reshape(-1)} tells Python to reshape the
array into a vector with as many elements as are in the array.
Mathematically, a vector is a one-dimensional array. In Python, the
difference is that a vector is indexed by a single number, while an
array is indexed by multiple numbers. After reshaping, we can look at
the number at index 0 of \texttt{t} using \texttt{t{[}0{]}}. If we don't
reshape first, we need to use \texttt{t{[}0,\ 0{]}} to get the same
result, so reshaping the array isn't required, but it is more
convenient. There is a nice explanation of array shapes
\href{https://stackoverflow.com/questions/22053050/difference-between-numpy-array-shape-r-1-and-r\#answer-22074424}{here}.

    The data file consists of three variables, which correspond to the LFP
data (\texttt{y}, in units of millivolts), the simultaneously recorded
spiking activity (\texttt{n}), and a time axis (\texttt{t}, in units of
seconds). Notice that the data are arrays, in which each row indicates a
separate trial, and each column indicates a point in time. In this case,
the variable \texttt{n} is binary; \texttt{n{[}k,i{]}=1} indicates a
spike in trial \texttt{k} at time index \texttt{i}.

    \textbf{Q.} What is the sampling frequency for these data?

\textbf{A.} We are given the time axis \texttt{t}. To compute the
sampling frequency, we compute the sampling interval:
\texttt{dt\ =\ t{[}2{]}-t{[}1{]}} and find \texttt{dt\ =\ 0.001}. The
sampling interval is therefore 1 ms, so the sampling frequency ( \(f\) )
is \(f = 1/dt = 1000\) Hz.

    The data for this trial are plotted in the Section \ref{fig11-1}. Visual
inspection immediately suggests that the LFP data exhibit a dominant
rhythm. By counting the number of peaks (or troughs) in 1 s of data, we
estimate the dominant rhythm to be \(\approx\) 10 Hz. However, careful
inspection suggests that other features appear in the LFP from this
first trial of data (i.e., additional lower-amplitude wiggles in the
signal). Let's keep this in mind as we continue the anal- ysis. To
visualize the spikes from the neuron, we plot the activity in the first
row of the data matrix (orange curve in the Section \ref{fig11-1}); this
is a crude representation of the activity but sufficient for the initial
inspection.

    \textbf{Q.} Continue your visual inspection for other trials of the
data. What do you observe?

    Visual inspection suggests that the neuron is active (i.e., it spikes)
during the trial. Of course, we may visualize and analyze features of
the spike train and the LFP using the methods described in earlier
chapters (e.g., \textbf{see problem 1}). However, our goal here is to
characterize the relation (if any) between the LFP and spikes. Let's
consider a relatively simple characterization of this relation, the
spike-triggered average.

    \subsubsection{Spike-Triggered Average}\label{spike-triggered-average}

    The \emph{spike-triggered average} (STA) is a relatively simple
procedure to visualize the relation between the LFP and spiking data. To
compute the STA, we implement the following procedure.

For each trial \(k = {1,...,K}\), do the following:

\begin{itemize}
\item
  Identify the time of each spike occurrence \(t_{k,i}\), where
  \(i \in \{1,...,N_k\}\), and \(N_{k}\) is the number of spikes in the
  \(k^{th}\) trial.
\item
  For each spike time \(t_{k,i}\), determine the LFP within a small
  temporal interval near the spike time \(LFP_{k,i}\).-
\item
  Average \(LFP_{k,i}\) across all spikes.
\end{itemize}

Despite this seemingly complicated procedure, the STA is a relatively
intuitive measure. The intuition is to find each spike and determine how
the LFP changes nearby. The procedure to compute the STA for each trial
is relatively straightforward to perform in Python:

    \begin{Verbatim}[commandchars=\\\{\}]
{\color{incolor}In [{\color{incolor}8}]:} \PY{n}{win} \PY{o}{=} \PY{l+m+mi}{100}
        \PY{n}{K} \PY{o}{=} \PY{n}{np}\PY{o}{.}\PY{n}{shape}\PY{p}{(}\PY{n}{n}\PY{p}{)}\PY{p}{[}\PY{l+m+mi}{0}\PY{p}{]}
        \PY{n}{N} \PY{o}{=} \PY{n}{np}\PY{o}{.}\PY{n}{shape}\PY{p}{(}\PY{n}{n}\PY{p}{)}\PY{p}{[}\PY{l+m+mi}{1}\PY{p}{]}
        \PY{n}{STA} \PY{o}{=} \PY{n}{np}\PY{o}{.}\PY{n}{zeros}\PY{p}{(}\PY{p}{[}\PY{n}{K}\PY{p}{,}\PY{l+m+mi}{2}\PY{o}{*}\PY{n}{win}\PY{o}{+}\PY{l+m+mi}{1}\PY{p}{]}\PY{p}{)}
        \PY{k}{for} \PY{n}{k} \PY{o+ow}{in} \PY{n}{np}\PY{o}{.}\PY{n}{arange}\PY{p}{(}\PY{n}{K}\PY{p}{)}\PY{p}{:}
            \PY{n}{spike\PYZus{}times} \PY{o}{=} \PY{n}{np}\PY{o}{.}\PY{n}{where}\PY{p}{(}\PY{n}{n}\PY{p}{[}\PY{n}{k}\PY{p}{,}\PY{p}{:}\PY{p}{]}\PY{o}{==}\PY{l+m+mi}{1}\PY{p}{)}
            \PY{n}{counter}\PY{o}{=}\PY{l+m+mi}{0}
            \PY{k}{for} \PY{n}{spike\PYZus{}t} \PY{o+ow}{in} \PY{n}{np}\PY{o}{.}\PY{n}{nditer}\PY{p}{(}\PY{n}{spike\PYZus{}times}\PY{p}{)}\PY{p}{:}
                \PY{k}{if} \PY{n}{win} \PY{o}{\PYZlt{}} \PY{n}{spike\PYZus{}t} \PY{o}{\PYZlt{}} \PY{n}{N}\PY{o}{\PYZhy{}}\PY{n}{win}\PY{o}{\PYZhy{}}\PY{l+m+mi}{1}\PY{p}{:}
                    \PY{n}{STA}\PY{p}{[}\PY{n}{k}\PY{p}{,}\PY{p}{:}\PY{p}{]} \PY{o}{=} \PY{n}{STA}\PY{p}{[}\PY{n}{k}\PY{p}{,}\PY{p}{:}\PY{p}{]} \PY{o}{+} \PY{n}{y}\PY{p}{[}\PY{n}{k}\PY{p}{,}\PY{n}{spike\PYZus{}t}\PY{o}{\PYZhy{}}\PY{n}{win}\PY{p}{:}\PY{n}{spike\PYZus{}t}\PY{o}{+}\PY{n}{win}\PY{o}{+}\PY{l+m+mi}{1}\PY{p}{]}
                    \PY{n}{counter} \PY{o}{+}\PY{o}{=} \PY{l+m+mi}{1}
            \PY{n}{STA}\PY{p}{[}\PY{n}{k}\PY{p}{,}\PY{p}{:}\PY{p}{]} \PY{o}{=} \PY{n}{STA}\PY{p}{[}\PY{n}{k}\PY{p}{,}\PY{p}{:}\PY{p}{]}\PY{o}{/}\PY{n}{counter}
\end{Verbatim}


    In this Python code, we must be careful to include only appropriate time
intervals when computing the STA.

    \textbf{Q.} Q: What is the purpose of the if-statement:

\texttt{if\ win\ \textless{}\ spike\_t\ \textless{}\ N-win-1:}

in the code?

    Notice that the variable \texttt{STA} is a matrix, with each row
corresponding to a separate trial. Let's plot the results for the STA in
four trials,

    \begin{Verbatim}[commandchars=\\\{\}]
{\color{incolor}In [{\color{incolor}12}]:} \PY{n}{dt} \PY{o}{=} \PY{n}{t}\PY{p}{[}\PY{l+m+mi}{1}\PY{p}{]}\PY{o}{\PYZhy{}}\PY{n}{t}\PY{p}{[}\PY{l+m+mi}{0}\PY{p}{]}
         \PY{n}{lags} \PY{o}{=} \PY{n}{np}\PY{o}{.}\PY{n}{arange}\PY{p}{(}\PY{o}{\PYZhy{}}\PY{n}{win}\PY{p}{,}\PY{n}{win}\PY{o}{+}\PY{l+m+mi}{1}\PY{p}{)}\PY{o}{*}\PY{n}{dt}   \PY{c+c1}{\PYZsh{} Make a time axis for plotting.}
         \PY{n}{plot}\PY{p}{(}\PY{n}{lags}\PY{p}{,} \PY{n}{STA}\PY{p}{[}\PY{l+m+mi}{0}\PY{p}{,}\PY{p}{:}\PY{p}{]}\PY{p}{)}              \PY{c+c1}{\PYZsh{} Show the STA for 4 trials.}
         \PY{n}{plot}\PY{p}{(}\PY{n}{lags}\PY{p}{,} \PY{n}{STA}\PY{p}{[}\PY{l+m+mi}{5}\PY{p}{,}\PY{p}{:}\PY{p}{]}\PY{p}{)}
         \PY{n}{plot}\PY{p}{(}\PY{n}{lags}\PY{p}{,} \PY{n}{STA}\PY{p}{[}\PY{l+m+mi}{9}\PY{p}{,}\PY{p}{:}\PY{p}{]}\PY{p}{)}
         \PY{n}{plot}\PY{p}{(}\PY{n}{lags}\PY{p}{,} \PY{n}{STA}\PY{p}{[}\PY{l+m+mi}{15}\PY{p}{,}\PY{p}{:}\PY{p}{]}\PY{p}{)}
         \PY{n}{xlabel}\PY{p}{(}\PY{l+s+s1}{\PYZsq{}}\PY{l+s+s1}{Time [ms]}\PY{l+s+s1}{\PYZsq{}}\PY{p}{)}
         \PY{n}{ylabel}\PY{p}{(}\PY{l+s+s1}{\PYZsq{}}\PY{l+s+s1}{Voltage [mV]}\PY{l+s+s1}{\PYZsq{}}\PY{p}{)}\PY{p}{;}
\end{Verbatim}


    \begin{center}
    \adjustimage{max size={0.9\linewidth}{0.9\paperheight}}{output_22_0.png}
    \end{center}
    { \hspace*{\fill} \\}
    
    And the STA results across all trials,

    \begin{Verbatim}[commandchars=\\\{\}]
{\color{incolor}In [{\color{incolor}14}]:} \PY{n}{plot}\PY{p}{(}\PY{n}{lags}\PY{p}{,}\PY{n}{np}\PY{o}{.}\PY{n}{transpose}\PY{p}{(}\PY{n}{STA}\PY{p}{)}\PY{p}{)}        \PY{c+c1}{\PYZsh{} Plot the STA results across all trials.}
         \PY{n}{xlabel}\PY{p}{(}\PY{l+s+s1}{\PYZsq{}}\PY{l+s+s1}{Time [ms]}\PY{l+s+s1}{\PYZsq{}}\PY{p}{)}
         \PY{n}{ylabel}\PY{p}{(}\PY{l+s+s1}{\PYZsq{}}\PY{l+s+s1}{Voltage [mV]}\PY{l+s+s1}{\PYZsq{}}\PY{p}{)}\PY{p}{;}
\end{Verbatim}


    \begin{center}
    \adjustimage{max size={0.9\linewidth}{0.9\paperheight}}{output_24_0.png}
    \end{center}
    { \hspace*{\fill} \\}
    
    The individual trial results suggest an approximate rhythmicity in the
STA; visual inspection reveals that the STA fluctuates with a period of
approximately 100 ms. However, these fluctuations are not phase-locked
across trials. For some trials, the LFP tends to be positive when the
cell spikes (i.e., at \(t = 0\) in the figure), while in other trials
the LFP tends to be negative when the cell spikes. The initial results
do not suggest a consistent relation exists between the spikes and the
LFP across trials.

However, let's not abandon all hope yet. We might be concerned that the
rhythmicity in the STA (Section \ref{pltsta}) is consistent with the
dominant rhythm of the LFP (Section \ref{figlfp_ex}). Because the STA is
an average of the LFP, we might expect the largest-amplitude features of
the LFP to make the biggest impact on the STA. Perhaps this
large-amplitude rhythm in the LFP is hiding more subtle features
embedded in lower-amplitude activity in the LFP. Let's continue the
search.

    \textbf{Q.} How would you update the preceding Python code to compute
both the average LFP (i.e., the STA) and the standard deviation of the
LFP across spikes for each trial?

    \subsubsection{Field-Triggered Average}\label{field-triggered-average}

    Let's now implement another visualization, the field-triggered average
(FTA). The FTA is similar in principle to the STA. However, for the FTA,
we use the field to organize the activity of the spikes (i.e., we use
the field to trigger the spikes). Here we choose a particular feature of
the field: the phase. The phase of neural signals is throught to play an
\href{https://www.ncbi.nlm.nih.gov/pmc/articles/PMC4605134/}{important
role in organizing brain activity}. Now we examine the role of the LFP
phase in organizing the spiking activity.

For each trial \(k = \{1, \ldots, K\}\), - Filter the LFP data in trial
\(k\) into a narrow band, and apply the Hilbert transform to estimate
the instantaneous phase.

\begin{itemize}
\tightlist
\item
  Sort the spike data in trial \(k\) according to the phase of the LFP.
\end{itemize}

For more information about the Hilbert transform and instantaneous
phase, check out this module discussing
\href{https://github.com/Mark-Kramer/Case-Studies-Python/tree/master/Cross-Frequency-Coupling}{cross-frequency
coupling}. We apply the same procedures here, but to a different end. To
implement computation of the FTA in Python, 

    \begin{Verbatim}[commandchars=\\\{\}]
{\color{incolor}In [{\color{incolor}20}]:} \PY{n}{dt} \PY{o}{=} \PY{n}{t}\PY{p}{[}\PY{l+m+mi}{1}\PY{p}{]}\PY{o}{\PYZhy{}}\PY{n}{t}\PY{p}{[}\PY{l+m+mi}{0}\PY{p}{]}                           \PY{c+c1}{\PYZsh{}Define the sampling interval.}
         \PY{n}{fNQ} \PY{o}{=} \PY{l+m+mi}{1}\PY{o}{/}\PY{n}{dt}\PY{o}{/}\PY{l+m+mi}{2}                             \PY{c+c1}{\PYZsh{}Define Nyquist frequency.}
         \PY{n}{Wn} \PY{o}{=} \PY{p}{[}\PY{l+m+mi}{9}\PY{p}{,}\PY{l+m+mi}{11}\PY{p}{]}                              \PY{c+c1}{\PYZsh{}Set the passband}
         \PY{n+nb}{ord}  \PY{o}{=} \PY{l+m+mi}{100}                               \PY{c+c1}{\PYZsh{}...and filter order,}
         \PY{n}{b} \PY{o}{=} \PY{n}{signal}\PY{o}{.}\PY{n}{firwin}\PY{p}{(}\PY{n+nb}{ord}\PY{p}{,} \PY{n}{Wn}\PY{p}{,} \PY{n}{nyq}\PY{o}{=}\PY{n}{fNQ}\PY{p}{,} \PY{n}{pass\PYZus{}zero}\PY{o}{=}\PY{k+kc}{False}\PY{p}{,} \PY{n}{window}\PY{o}{=}\PY{l+s+s1}{\PYZsq{}}\PY{l+s+s1}{hamming}\PY{l+s+s1}{\PYZsq{}}\PY{p}{)}\PY{p}{;} \PY{c+c1}{\PYZsh{}...build bandpass filter.}
         \PY{n}{FTA}\PY{o}{=}\PY{n}{np}\PY{o}{.}\PY{n}{zeros}\PY{p}{(}\PY{p}{[}\PY{n}{K}\PY{p}{,}\PY{n}{N}\PY{p}{]}\PY{p}{)}                      \PY{c+c1}{\PYZsh{}Create a variable to hold the FTA.}
         \PY{k}{for} \PY{n}{k} \PY{o+ow}{in} \PY{n}{np}\PY{o}{.}\PY{n}{arange}\PY{p}{(}\PY{n}{K}\PY{p}{)}\PY{p}{:}                   \PY{c+c1}{\PYZsh{}For each trial,}
             \PY{n}{Vlo} \PY{o}{=} \PY{n}{signal}\PY{o}{.}\PY{n}{filtfilt}\PY{p}{(}\PY{n}{b}\PY{p}{,} \PY{l+m+mi}{1}\PY{p}{,} \PY{n}{y}\PY{p}{[}\PY{n}{k}\PY{p}{,}\PY{p}{:}\PY{p}{]}\PY{p}{)}  \PY{c+c1}{\PYZsh{} ... and apply filter.}
             \PY{n}{phi} \PY{o}{=} \PY{n}{np}\PY{o}{.}\PY{n}{angle}\PY{p}{(}\PY{n}{signal}\PY{o}{.}\PY{n}{hilbert}\PY{p}{(}\PY{n}{Vlo}\PY{p}{)}\PY{p}{)}  \PY{c+c1}{\PYZsh{} Compute phase of low\PYZhy{}freq signal}
             \PY{n}{indices} \PY{o}{=} \PY{n}{np}\PY{o}{.}\PY{n}{argsort}\PY{p}{(}\PY{n}{phi}\PY{p}{)}            \PY{c+c1}{\PYZsh{}... get indices of sorted phase,}
             \PY{n}{FTA}\PY{p}{[}\PY{n}{k}\PY{p}{,}\PY{p}{:}\PY{p}{]} \PY{o}{=} \PY{n}{n}\PY{p}{[}\PY{n}{k}\PY{p}{,}\PY{n}{indices}\PY{p}{]}              \PY{c+c1}{\PYZsh{}... and store the sorted spikes.}
         
         \PY{c+c1}{\PYZsh{}Plot the average FTA versus phase.}
         \PY{n}{plot}\PY{p}{(}\PY{n}{np}\PY{o}{.}\PY{n}{linspace}\PY{p}{(}\PY{o}{\PYZhy{}}\PY{n}{np}\PY{o}{.}\PY{n}{pi}\PY{p}{,}\PY{n}{np}\PY{o}{.}\PY{n}{pi}\PY{p}{,}\PY{n}{N}\PY{p}{)}\PY{p}{,} \PY{n}{np}\PY{o}{.}\PY{n}{mean}\PY{p}{(}\PY{n}{FTA}\PY{p}{,}\PY{l+m+mi}{0}\PY{p}{)}\PY{p}{)}
         \PY{n}{xlabel}\PY{p}{(}\PY{l+s+s1}{\PYZsq{}}\PY{l+s+s1}{Phase}\PY{l+s+s1}{\PYZsq{}}\PY{p}{)}
         \PY{n}{ylabel}\PY{p}{(}\PY{l+s+s1}{\PYZsq{}}\PY{l+s+s1}{FTA}\PY{l+s+s1}{\PYZsq{}}\PY{p}{)}\PY{p}{;}
\end{Verbatim}


    \begin{center}
    \adjustimage{max size={0.9\linewidth}{0.9\paperheight}}{output_29_0.png}
    \end{center}
    { \hspace*{\fill} \\}
    
    Notice the steps to set up the filter. We choose a bandpass filter from
9--11 Hz. We choose this interval to focus on the LFP rhythm of largest
amplitude (\(\approx\) 10 Hz), which we identified through visual
inspection (Section \ref{figlfp_ex}). For each trial, we apply the
filter to the LFP and then use the Hilbert transform
(\texttt{signal.hilbert}) to estimate the phase. Finally, we sort this
phase and use the sorted indices to arrange the spikes and store the
results. We show the FTA averaged across all trials in the
Section \ref{figfta}. In this case, no modulation in the number of
spikes is apparent across trials. Instead, the number of spikes at each
phase appears equally likely.

We may apply the FTA analysis to different frequency intervals of the
LFP. Choosing a frequency interval may be motivated by our knowledge of
the neural system generating the activity or by inspection of the field
and spiking data.

    \textbf{Q.} Investigate different frequency bands in the FTA analysis.
Do you observe any interesting features?

\(Hint\): Consider frequencies near 45 Hz.

    One final note about the FTA. The purpose of this measure is
visualization, not statistical testing. Hopefully, this visual
inspection will provide some insight into the data and guide continuing
studies in promising directions. In what follows, we consider approaches
to test for significant effects when we build a GLM to assess
spike-field relations.

    \subsubsection{Spike-field coherence}\label{spike-field-coherence}

    To characterize the relation between the LFP and spikes, we have so far
visualized the data and computed relatively simple and intuitive aids to
visualization. Now we examine a more sophisticated and powerful method:
the spike-field coherence. It's common to investigate the coherence
applied to field activity; we may refer to this type of coherence as
field-field coherence to distinguish it from spike-field coherence of
interest here. In practice, this distinction is usually unnecessary, as
in most cases the context is clear. However, in this chapter, we are
careful to distinguish field-field coherence from spike-field coherence.

The field-field coherence is a frequency domain measure of linear
association between two continuous time series. Note that, in practice,
we observe a sampled version of a presumably continuous signal. This
sampling impacts aspects of our data analysis, for example spectral
estimators (see
\href{../Analysis\%20of\%20Rhythmic\%20Activity\%20in\%20the\%20Scalp\%20EEG/Analysis\%20of\%20rhythmic\%20activity\%20in\%20the\%20Scalp\%20EEG.ipynb}{Analysis
of Rhythmic Activity of the Scalp EEG} and
\href{../Analysis\%20of\%20rhythmic\%20activity\%20in\%20the\%20Electrocorticogram/Analysis\%20of\%20rhythmic\%20activity\%20in\%20the\%20Electrocorticogram.ipynb}{Analysis
of Rhythmic Activity in the Electrocorticogram}. We showed in
\href{to\%20add}{REF} that two fields are coherent across trials at
frequency \(f_0\) if the fields possess a constant phase relation across
trials at that frequency. The same relation holds for the spike-field
coherence. However, differences arise because of the point process
nature of the spike train data. These differences have profound
implications with dangerous consequences. In this chapter, we explore
some of these issues. For a deeper mathematical discussion and potential
solutions, see
\href{https://www.ncbi.nlm.nih.gov/pubmed/21671792}{Lepage et al, 2011}
and \href{https://www.ncbi.nlm.nih.gov/pmc/articles/PMC3800189/}{Lepage
et al, 2013}.

    \paragraph{Mathematical Description of spike-field
coherence.}\label{mathematical-description-of-spike-field-coherence.}

Let's begin with a mathematical description of the spike-field
coherence. To do so, we need to introduce some notation, which is
identical to that used in earlier chapters, but we include it here for
completeness. A more detailed description may be found in
\href{https://www.ncbi.nlm.nih.gov/pubmed/21671792}{Lepage et al, 2011}.

We considered spectral estimators for a field in
\href{../Analysis\%20of\%20Rhythmic\%20Activity\%20in\%20the\%20Scalp\%20EEG/Analysis\%20of\%20rhythmic\%20activity\%20in\%20the\%20Scalp\%20EEG.ipynb}{Analysis
of Rhythmic Activity of the Scalp EEG} and for a point process in
\href{to\%20add}{REF}. We restate the Fourier transform for a time
series \(x\),

\[
X_j = \sum_{n=1}^N x_n \exp(-2 \pi i \, f_j \, t_n)
\]

where \(x_n\) is the signal at time index
\(t_n = dt \{1, 2, 3, . . . N\}\), and the frequencies \(f_j = j/T\),
where \(j=\{-N/2+1, -N/2+2, \ldots , N/2-1, N/2\}\). The spectral
density of the time series is then,

\[
S_{xx,j} = \frac{2 dt^2}{T} X_j X^*_j .
\]

Here, the time series can be either a field (i.e., the LFP) or a point
process (i.e., the spike train). Notice that we employ the same
mathematical formula to compute the spectrum for each time series.

    For the spike train data, we first subtract the mean or expected number
of spikes in each time interval and then apply the Fourier transform. In
other words, the signal is the \emph{centered increments} (see
\href{to\%20add}{chapter 8}).

    Then, to estimate the coherence between two time series \(x\) and \(y\),

\[
\kappa_{xy,j} = \frac{ \mid <S_{xy,j}> \mid }{ \sqrt{<S_{xx,j}>} \sqrt{<S_{yy,j}>}}
\label{eq:SF_k0}
\]

where \(\mid <S_{xy,j}> \mid\) indicates the magnitude of the trial
averaged cross spectrum, and \(\mid <S_{xx,j}> \mid\) and
\(\mid <S_{yy,j}> \mid\) indicate the magnitude of the trial averaged
spectra of \(x\) and \(y\), respectively. So far, there's nothing new
here; we've just restated the standard expressions for the spectrum and
coherence. To compute the spike-field cohernece, we simply interpret one
of the time series as a point process. To make this more obvious in our
mathematical expression, we replace \(x\) in the equation above with the
symbol \(n\), as a reminder that this time series represents the
"number" of spikes,

\[
\kappa_{ny,j} = \frac{ \mid <S_{ny,j}> \mid }{ \sqrt{<S_{nn,j}>} \sqrt{<S_{yy,j}>}}
\label{eq:SF_k}
\]

    In the equation above, the numerator is now the magnitude of the trial
averaged cross spectrum between the field \(y\) and spikes \(n\), and
the denominator contains the trial averaged spectrum of the spike \(n\)
and the trial averaged spectrum of the field \(y\).

    We could instead write the \emph{sample} coherence, because this
equation uses the observed data to estimate the theoretical coherence
that we would see if we were to keep repeating this experiment. This
distinction is not essential to our goals here, but is important when
talking to your statistics-minded colleagues. Throughout this chapter
and others, we omit the term "sample" when referring to sample means,
variances, covariances, spectra, and so forth, unless this distinction
becomes essential to our discussion.

    \subsubsection{Computing the Spike-Field-Coherence in
MATLAB.}\label{computing-the-spike-field-coherence-in-matlab.}

As discussed in other modules
(\href{../Analysis\%20of\%20Rhythmic\%20Activity\%20in\%20the\%20Scalp\%20EEG/Analysis\%20of\%20rhythmic\%20activity\%20in\%20the\%20Scalp\%20EEG.ipynb}{Analysis
of Rhythmic Activity of the Scalp EEG} and
\href{../Analysis\%20of\%20rhythmic\%20activity\%20in\%20the\%20Electrocorticogram/Analysis\%20of\%20rhythmic\%20activity\%20in\%20the\%20Electrocorticogram.ipynb}{Analysis
of Rhythmic Activity in the Electrocorticogram}, many issues are
involved in spectral analysis, for example, the notions of tapering.
These important issues apply for the computation of spike-field
coherence as well. In practice, multitaper methods are often used to
compute the spike-field coherence. In what follows, we simply a simple
tapering approach to the field data.

Let's now compute the spike-field coherence for the data of interest
here. It's relatively straightforward to do so in Python:

    \begin{Verbatim}[commandchars=\\\{\}]
{\color{incolor}In [{\color{incolor}21}]:} \PY{n}{SYY} \PY{o}{=} \PY{n}{np}\PY{o}{.}\PY{n}{zeros}\PY{p}{(}\PY{n+nb}{int}\PY{p}{(}\PY{n}{N}\PY{o}{/}\PY{l+m+mi}{2}\PY{o}{+}\PY{l+m+mi}{1}\PY{p}{)}\PY{p}{)}
         \PY{n}{SNN} \PY{o}{=} \PY{n}{np}\PY{o}{.}\PY{n}{zeros}\PY{p}{(}\PY{n+nb}{int}\PY{p}{(}\PY{n}{N}\PY{o}{/}\PY{l+m+mi}{2}\PY{o}{+}\PY{l+m+mi}{1}\PY{p}{)}\PY{p}{)}
         \PY{n}{SYN} \PY{o}{=} \PY{n}{np}\PY{o}{.}\PY{n}{zeros}\PY{p}{(}\PY{n+nb}{int}\PY{p}{(}\PY{n}{N}\PY{o}{/}\PY{l+m+mi}{2}\PY{o}{+}\PY{l+m+mi}{1}\PY{p}{)}\PY{p}{,} \PY{n}{dtype}\PY{o}{=}\PY{n+nb}{complex}\PY{p}{)}
         
         \PY{k}{for} \PY{n}{k} \PY{o+ow}{in} \PY{n}{np}\PY{o}{.}\PY{n}{arange}\PY{p}{(}\PY{n}{K}\PY{p}{)}\PY{p}{:}
             \PY{n}{yf} \PY{o}{=} \PY{n}{np}\PY{o}{.}\PY{n}{fft}\PY{o}{.}\PY{n}{rfft}\PY{p}{(}\PY{p}{(}\PY{n}{y}\PY{p}{[}\PY{n}{k}\PY{p}{,}\PY{p}{:}\PY{p}{]}\PY{o}{\PYZhy{}}\PY{n}{np}\PY{o}{.}\PY{n}{mean}\PY{p}{(}\PY{n}{y}\PY{p}{[}\PY{n}{k}\PY{p}{,}\PY{p}{:}\PY{p}{]}\PY{p}{)}\PY{p}{)} \PY{o}{*}\PY{n}{np}\PY{o}{.}\PY{n}{hanning}\PY{p}{(}\PY{n}{N}\PY{p}{)}\PY{p}{)}    \PY{c+c1}{\PYZsh{} Hanning taper the field,}
             \PY{n}{nf} \PY{o}{=} \PY{n}{np}\PY{o}{.}\PY{n}{fft}\PY{o}{.}\PY{n}{rfft}\PY{p}{(}\PY{p}{(}\PY{n}{n}\PY{p}{[}\PY{n}{k}\PY{p}{,}\PY{p}{:}\PY{p}{]}\PY{o}{\PYZhy{}}\PY{n}{np}\PY{o}{.}\PY{n}{mean}\PY{p}{(}\PY{n}{n}\PY{p}{[}\PY{n}{k}\PY{p}{,}\PY{p}{:}\PY{p}{]}\PY{p}{)}\PY{p}{)}\PY{p}{)}                   \PY{c+c1}{\PYZsh{} ... but do not taper the spikes.}
             \PY{n}{SYY} \PY{o}{=} \PY{n}{SYY} \PY{o}{+} \PY{p}{(} \PY{n}{np}\PY{o}{.}\PY{n}{real}\PY{p}{(} \PY{n}{yf}\PY{o}{*}\PY{n}{np}\PY{o}{.}\PY{n}{conj}\PY{p}{(}\PY{n}{yf}\PY{p}{)} \PY{p}{)} \PY{p}{)}\PY{o}{/}\PY{n}{K}                  \PY{c+c1}{\PYZsh{} Field spectrum}
             \PY{n}{SNN} \PY{o}{=} \PY{n}{SNN} \PY{o}{+} \PY{p}{(} \PY{n}{np}\PY{o}{.}\PY{n}{real}\PY{p}{(} \PY{n}{nf}\PY{o}{*}\PY{n}{np}\PY{o}{.}\PY{n}{conj}\PY{p}{(}\PY{n}{nf}\PY{p}{)} \PY{p}{)} \PY{p}{)}\PY{o}{/}\PY{n}{K}                  \PY{c+c1}{\PYZsh{} Spike spectrum}
             \PY{n}{SYN} \PY{o}{=} \PY{n}{SYN} \PY{o}{+} \PY{p}{(}          \PY{n}{yf}\PY{o}{*}\PY{n}{np}\PY{o}{.}\PY{n}{conj}\PY{p}{(}\PY{n}{nf}\PY{p}{)}   \PY{p}{)}\PY{o}{/}\PY{n}{K}                  \PY{c+c1}{\PYZsh{} Cross spectrum}
         
         \PY{n}{cohr} \PY{o}{=} \PY{n}{np}\PY{o}{.}\PY{n}{real}\PY{p}{(}\PY{n}{SYN}\PY{o}{*}\PY{n}{np}\PY{o}{.}\PY{n}{conj}\PY{p}{(}\PY{n}{SYN}\PY{p}{)}\PY{p}{)} \PY{o}{/} \PY{n}{SYY} \PY{o}{/} \PY{n}{SNN}                     \PY{c+c1}{\PYZsh{} Coherence}
         \PY{n}{f} \PY{o}{=} \PY{n}{np}\PY{o}{.}\PY{n}{fft}\PY{o}{.}\PY{n}{rfftfreq}\PY{p}{(}\PY{n}{N}\PY{p}{,} \PY{n}{dt}\PY{p}{)}                                       \PY{c+c1}{\PYZsh{} Frequency axis for plotting}
\end{Verbatim}


    Inside of the \texttt{for} statement, we first compute the Fourier
transform of the field (\texttt{yf}) and the spikes (\texttt{nf}) for
trial \texttt{k}. Notice that we subtract the mean from each signal
before computing the Fourier transform, and that we apply a Hanning
taper to the field data. We estimate the spectra for the field
(\texttt{SYY}) and the spikes (\texttt{SNN}), and the cross spectrum
(\texttt{SYN}) averaged across all trials. We then compute the coherence
(\texttt{cohr}) and define a frequency axis to plot the results
(\texttt{f}).

Let's now display the results,

    \begin{Verbatim}[commandchars=\\\{\}]
{\color{incolor}In [{\color{incolor}23}]:} \PY{n}{plt}\PY{o}{.}\PY{n}{subplot}\PY{p}{(}\PY{l+m+mi}{1}\PY{p}{,}\PY{l+m+mi}{3}\PY{p}{,}\PY{l+m+mi}{1}\PY{p}{)}         \PY{c+c1}{\PYZsh{} Plot the spike spectrum.}
         \PY{n}{plot}\PY{p}{(}\PY{n}{f}\PY{p}{,}\PY{n}{SNN}\PY{p}{)}
         \PY{n}{plt}\PY{o}{.}\PY{n}{xlim}\PY{p}{(}\PY{p}{[}\PY{l+m+mi}{0}\PY{p}{,} \PY{l+m+mi}{100}\PY{p}{]}\PY{p}{)}
         \PY{n}{xlabel}\PY{p}{(}\PY{l+s+s1}{\PYZsq{}}\PY{l+s+s1}{Frequency [Hz]}\PY{l+s+s1}{\PYZsq{}}\PY{p}{)}
         \PY{n}{ylabel}\PY{p}{(}\PY{l+s+s1}{\PYZsq{}}\PY{l+s+s1}{Power [Hz]}\PY{l+s+s1}{\PYZsq{}}\PY{p}{)}
         \PY{n}{title}\PY{p}{(}\PY{l+s+s1}{\PYZsq{}}\PY{l+s+s1}{SNN}\PY{l+s+s1}{\PYZsq{}}\PY{p}{)}
         
         \PY{n}{plt}\PY{o}{.}\PY{n}{subplot}\PY{p}{(}\PY{l+m+mi}{1}\PY{p}{,}\PY{l+m+mi}{3}\PY{p}{,}\PY{l+m+mi}{2}\PY{p}{)}        \PY{c+c1}{\PYZsh{} Plot the field spectrum,}
         \PY{n}{T} \PY{o}{=} \PY{n}{t}\PY{p}{[}\PY{o}{\PYZhy{}}\PY{l+m+mi}{1}\PY{p}{]}
         \PY{n}{plot}\PY{p}{(}\PY{n}{f}\PY{p}{,}\PY{n}{dt}\PY{o}{*}\PY{o}{*}\PY{l+m+mi}{2}\PY{o}{/}\PY{n}{T}\PY{o}{*}\PY{n}{SYY}\PY{p}{)}       \PY{c+c1}{\PYZsh{} ... with the standard scaling.}
         \PY{n}{plt}\PY{o}{.}\PY{n}{xlim}\PY{p}{(}\PY{p}{[}\PY{l+m+mi}{0}\PY{p}{,} \PY{l+m+mi}{100}\PY{p}{]}\PY{p}{)}
         \PY{n}{xlabel}\PY{p}{(}\PY{l+s+s1}{\PYZsq{}}\PY{l+s+s1}{Frequency [Hz]}\PY{l+s+s1}{\PYZsq{}}\PY{p}{)}
         \PY{n}{ylabel}\PY{p}{(}\PY{l+s+s1}{\PYZsq{}}\PY{l+s+s1}{Power [Hz]}\PY{l+s+s1}{\PYZsq{}}\PY{p}{)}
         \PY{n}{title}\PY{p}{(}\PY{l+s+s1}{\PYZsq{}}\PY{l+s+s1}{SYY}\PY{l+s+s1}{\PYZsq{}}\PY{p}{)}
         
         \PY{n}{plt}\PY{o}{.}\PY{n}{subplot}\PY{p}{(}\PY{l+m+mi}{1}\PY{p}{,}\PY{l+m+mi}{3}\PY{p}{,}\PY{l+m+mi}{3}\PY{p}{)}        \PY{c+c1}{\PYZsh{} Plot the coherence}
         \PY{n}{plot}\PY{p}{(}\PY{n}{f}\PY{p}{,}\PY{n}{cohr}\PY{p}{)}
         \PY{n}{plt}\PY{o}{.}\PY{n}{xlim}\PY{p}{(}\PY{p}{[}\PY{l+m+mi}{0}\PY{p}{,} \PY{l+m+mi}{100}\PY{p}{]}\PY{p}{)}
         \PY{n}{plt}\PY{o}{.}\PY{n}{ylim}\PY{p}{(}\PY{p}{[}\PY{l+m+mi}{0}\PY{p}{,} \PY{l+m+mi}{1}\PY{p}{]}\PY{p}{)}
         \PY{n}{xlabel}\PY{p}{(}\PY{l+s+s1}{\PYZsq{}}\PY{l+s+s1}{Frequency [Hz]}\PY{l+s+s1}{\PYZsq{}}\PY{p}{)}
         \PY{n}{ylabel}\PY{p}{(}\PY{l+s+s1}{\PYZsq{}}\PY{l+s+s1}{Coherence}\PY{l+s+s1}{\PYZsq{}}\PY{p}{)}\PY{p}{;}
\end{Verbatim}


    \begin{center}
    \adjustimage{max size={0.9\linewidth}{0.9\paperheight}}{output_43_0.png}
    \end{center}
    { \hspace*{\fill} \\}
    
    \textbf{Q:} Consider the spike spectrum, \texttt{Snn}, plotted in the
figure above. What are the dominant rhythms? At frequencies beyond these
dominant rhythms, the spectrum appears to fluctuate around a constant
value. What is this constant value?

\textbf{A.} To answer the first question, we determine through visual
inspection of the figure that the dominant rhythm (i.e., the frequency
with the most power) occurs at 10 Hz. We also note the presence of a
second peak near 45 Hz.

To answer the second question, we note that the spike spectrum
asymptotes at the expected spike rate (see \href{add\%20ref}{chapter
10}). For these data, we can estimate the expected spike rate as

\texttt{firing\_rate\ =\ np.mean(np.sum(n,1))/(N*dt)}

Computing this quantity in Python, we find an expected spike rate of
approximately 89 Hz, consistent with the high-frequency behavior of
\texttt{Snn} plotted in the figure.

    \textbf{Q:} onsider the field spectrum, \texttt{Syy}, plotted in the
figure above. What are the dominant rhythms? Do you observe any other
interesting features in this spectrum?

\textbf{A:} Visual inspection of the figure reveals that the dominant
rhythm occurs at 10 Hz. At first glance, no additional spectral features
stand out.

    These observations of the spike spectrum and field spectrum reveal that
both signals exhibit rhythmic activity at 10 Hz. Therefore, a reasonable
place to look for spike-field coherence is near 10 Hz, where both the
spikes and the field are rhythmic. However, visual inspection of the
spike-field coherence does not indicate coherence at this frequency.
Instead, we find a large peak in the spike-field coherence at 45 Hz.
Identifying this strong coherence at 45 Hz suggests that we reexamine
the spectra. Indeed, careful inspection of the spike spectrum and field
spectrum does suggest rhythmic activity at 45 Hz.

    \textbf{Q.} What is the approximate value of the \emph{imaginary} part
of \(X_j\) for \(f_j = 10\) Hz? \emph{Hint}: Consider the plot of the
product of the sine function and the data.

    \textbf{Q:} Consider the field spectrum on a decibel scale (see
\href{../Analysis\%20of\%20Rhythmic\%20Activity\%20in\%20the\%20Scalp\%20EEG/Analysis\%20of\%20rhythmic\%20activity\%20in\%20the\%20Scalp\%20EEG.ipynb\#decibel-scaling}{decibel
scaling in Analysis of Rhythmic Activity of the Scalp EEG}). What
rhythms do you observe?

    \textbf{Q:} Compare the results of your spike-field coherence analysis
with the FTA plotted in Section \ref{figfta}. How does the peak in the
spike-field coherence relate to interesting structure in the FTA?

    The spike-field coherence results again reveal an important feature of
coherence analysis. Two signals with high power at the same frequency
are not necessarily coherent at this frequency; two signals may possess
rhythmic activity at the same frequency, but these rhythms may not
coordinate across trials. Conversely, two signals with low power at the
same frequency may have strong coherence at that frequency; although the
rhythm is weak, the two signals may still coordinate activity across
trials at this frequency. These notions apply both to spike-field
coherence and field-field coherence (the latter illustrated in
\href{to\%20add}{chapter 4}).

The multitaper method to compute the spike-field coherence is a powerful
tool in our data analysis arsenal. There's much more to say about this
approach, and interested readers are directed to
\href{https://www.ncbi.nlm.nih.gov/pubmed/11255566}{Jarvis and Mitra,
2001}, \href{https://www.ncbi.nlm.nih.gov/pubmed/21671792}{Lepage et al,
2011} and
\href{https://www.ncbi.nlm.nih.gov/pmc/articles/PMC3800189/}{Lepage et
al, 2013}.

    \subsubsection{The Impact of Firing Rate on the Spike-Field
Coherence}\label{the-impact-of-firing-rate-on-the-spike-field-coherence}

Often, in the analysis of neural data, we compare the coherence between
two pairs of signals. For example, in analysis of scalp EEG data, we
might compare the coherence between voltage activity recorded at
electrodes A and B with the coherence between voltage activity recorded
at electrodes A and C. If we find that electrodes A and B have higher
coherence at some frequency than electrodes A and C, we may conclude
that the two brain regions A and B coordinate more strongly at this
frequency. In this thought experiment, we are comparing the field-field
coherence, which is not affected by the amplitude of the signals. For
example, if we multiply the amplitude of signal C by a factor of 0.1,
the field-field coherence does not change. To gain some intuition for
this result, note that in the computation of the coherence
(Section \ref{eqfield-field-coherence}), we divide by the spectrum of
each signal. In this way, a multiplicative change in signal amplitude
appears in the numerator and denominator of the coherence formula and
therefore (in this case) factors out.

We might expect the same for spike-field coherence. To test this, let's
manipulate the experimental data provided by our collaborator. We scale
the field data by a factor of 0.1 and recompute the spike-field
coherence. Scaling the field data is easy to do in Python:

    \begin{Verbatim}[commandchars=\\\{\}]
{\color{incolor}In [{\color{incolor}28}]:} \PY{n}{y\PYZus{}scaled} \PY{o}{=} \PY{l+m+mf}{0.1}\PY{o}{*}\PY{n}{y}
\end{Verbatim}


    With this change in the LFP data (\texttt{y}), we now recompute the
spike-field coherence. To do so, let's first define a function to
compute the spike-field coherence,

    \begin{Verbatim}[commandchars=\\\{\}]
{\color{incolor}In [{\color{incolor}29}]:} \PY{k}{def} \PY{n+nf}{coherence}\PY{p}{(}\PY{n}{n}\PY{p}{,}\PY{n}{y}\PY{p}{,}\PY{n}{t}\PY{p}{)}\PY{p}{:}                           \PY{c+c1}{\PYZsh{}INPUT (spikes, fields, time)}
             \PY{n}{K} \PY{o}{=} \PY{n}{np}\PY{o}{.}\PY{n}{shape}\PY{p}{(}\PY{n}{n}\PY{p}{)}\PY{p}{[}\PY{l+m+mi}{0}\PY{p}{]}                          \PY{c+c1}{\PYZsh{}... where spikes and fields are arrays [trials, time]}
             \PY{n}{N} \PY{o}{=} \PY{n}{np}\PY{o}{.}\PY{n}{shape}\PY{p}{(}\PY{n}{n}\PY{p}{)}\PY{p}{[}\PY{l+m+mi}{1}\PY{p}{]}
             \PY{n}{T} \PY{o}{=} \PY{n}{t}\PY{p}{[}\PY{o}{\PYZhy{}}\PY{l+m+mi}{1}\PY{p}{]}
             \PY{n}{SYY} \PY{o}{=} \PY{n}{np}\PY{o}{.}\PY{n}{zeros}\PY{p}{(}\PY{n+nb}{int}\PY{p}{(}\PY{n}{N}\PY{o}{/}\PY{l+m+mi}{2}\PY{o}{+}\PY{l+m+mi}{1}\PY{p}{)}\PY{p}{)}
             \PY{n}{SNN} \PY{o}{=} \PY{n}{np}\PY{o}{.}\PY{n}{zeros}\PY{p}{(}\PY{n+nb}{int}\PY{p}{(}\PY{n}{N}\PY{o}{/}\PY{l+m+mi}{2}\PY{o}{+}\PY{l+m+mi}{1}\PY{p}{)}\PY{p}{)}
             \PY{n}{SYN} \PY{o}{=} \PY{n}{np}\PY{o}{.}\PY{n}{zeros}\PY{p}{(}\PY{n+nb}{int}\PY{p}{(}\PY{n}{N}\PY{o}{/}\PY{l+m+mi}{2}\PY{o}{+}\PY{l+m+mi}{1}\PY{p}{)}\PY{p}{,} \PY{n}{dtype}\PY{o}{=}\PY{n+nb}{complex}\PY{p}{)}
             
             \PY{k}{for} \PY{n}{k} \PY{o+ow}{in} \PY{n}{np}\PY{o}{.}\PY{n}{arange}\PY{p}{(}\PY{n}{K}\PY{p}{)}\PY{p}{:}
                 \PY{n}{yf} \PY{o}{=} \PY{n}{np}\PY{o}{.}\PY{n}{fft}\PY{o}{.}\PY{n}{rfft}\PY{p}{(}\PY{p}{(}\PY{n}{y}\PY{p}{[}\PY{n}{k}\PY{p}{,}\PY{p}{:}\PY{p}{]}\PY{o}{\PYZhy{}}\PY{n}{np}\PY{o}{.}\PY{n}{mean}\PY{p}{(}\PY{n}{y}\PY{p}{[}\PY{n}{k}\PY{p}{,}\PY{p}{:}\PY{p}{]}\PY{p}{)}\PY{p}{)} \PY{o}{*}\PY{n}{np}\PY{o}{.}\PY{n}{hanning}\PY{p}{(}\PY{n}{N}\PY{p}{)}\PY{p}{)}    \PY{c+c1}{\PYZsh{} Hanning taper the field,}
                 \PY{n}{nf} \PY{o}{=} \PY{n}{np}\PY{o}{.}\PY{n}{fft}\PY{o}{.}\PY{n}{rfft}\PY{p}{(}\PY{p}{(}\PY{n}{n}\PY{p}{[}\PY{n}{k}\PY{p}{,}\PY{p}{:}\PY{p}{]}\PY{o}{\PYZhy{}}\PY{n}{np}\PY{o}{.}\PY{n}{mean}\PY{p}{(}\PY{n}{n}\PY{p}{[}\PY{n}{k}\PY{p}{,}\PY{p}{:}\PY{p}{]}\PY{p}{)}\PY{p}{)}\PY{p}{)}                   \PY{c+c1}{\PYZsh{} ... but do not taper the spikes.}
                 \PY{n}{SYY} \PY{o}{=} \PY{n}{SYY} \PY{o}{+} \PY{p}{(} \PY{n}{np}\PY{o}{.}\PY{n}{real}\PY{p}{(} \PY{n}{yf}\PY{o}{*}\PY{n}{np}\PY{o}{.}\PY{n}{conj}\PY{p}{(}\PY{n}{yf}\PY{p}{)} \PY{p}{)} \PY{p}{)}\PY{o}{/}\PY{n}{K}                  \PY{c+c1}{\PYZsh{} Field spectrum}
                 \PY{n}{SNN} \PY{o}{=} \PY{n}{SNN} \PY{o}{+} \PY{p}{(} \PY{n}{np}\PY{o}{.}\PY{n}{real}\PY{p}{(} \PY{n}{nf}\PY{o}{*}\PY{n}{np}\PY{o}{.}\PY{n}{conj}\PY{p}{(}\PY{n}{nf}\PY{p}{)} \PY{p}{)} \PY{p}{)}\PY{o}{/}\PY{n}{K}                  \PY{c+c1}{\PYZsh{} Spike spectrum}
                 \PY{n}{SYN} \PY{o}{=} \PY{n}{SYN} \PY{o}{+} \PY{p}{(}          \PY{n}{yf}\PY{o}{*}\PY{n}{np}\PY{o}{.}\PY{n}{conj}\PY{p}{(}\PY{n}{nf}\PY{p}{)}   \PY{p}{)}\PY{o}{/}\PY{n}{K}                  \PY{c+c1}{\PYZsh{} Cross spectrum}
         
             \PY{n}{cohr} \PY{o}{=} \PY{n}{np}\PY{o}{.}\PY{n}{real}\PY{p}{(}\PY{n}{SYN}\PY{o}{*}\PY{n}{np}\PY{o}{.}\PY{n}{conj}\PY{p}{(}\PY{n}{SYN}\PY{p}{)}\PY{p}{)} \PY{o}{/} \PY{n}{SYY} \PY{o}{/} \PY{n}{SNN}                     \PY{c+c1}{\PYZsh{} Coherence}
             \PY{n}{f} \PY{o}{=} \PY{n}{np}\PY{o}{.}\PY{n}{fft}\PY{o}{.}\PY{n}{rfftfreq}\PY{p}{(}\PY{n}{N}\PY{p}{,} \PY{n}{dt}\PY{p}{)}                                       \PY{c+c1}{\PYZsh{} Frequency axis for plotting}
             
             \PY{k}{return} \PY{p}{(}\PY{n}{cohr}\PY{p}{,} \PY{n}{f}\PY{p}{,} \PY{n}{SYY}\PY{p}{,} \PY{n}{SNN}\PY{p}{,} \PY{n}{SYN}\PY{p}{)}
\end{Verbatim}


    Now, with the fucntion \texttt{coherence} defined, let's examine how a
multiplicative change in the field \texttt{y} impacts the spike-field
coherence,

    \begin{Verbatim}[commandchars=\\\{\}]
{\color{incolor}In [{\color{incolor}39}]:} \PY{p}{[}\PY{n}{cohr}\PY{p}{,} \PY{n}{f}\PY{p}{,} \PY{n}{SYY}\PY{p}{,} \PY{n}{SNN}\PY{p}{,} \PY{n}{SYN}\PY{p}{]} \PY{o}{=} \PY{n}{coherence}\PY{p}{(}\PY{n}{n}\PY{p}{,}\PY{n}{y}\PY{p}{,}\PY{n}{t}\PY{p}{)}             \PY{c+c1}{\PYZsh{} Compute spike\PYZhy{}field cohernece with original y.}
         \PY{n}{plot}\PY{p}{(}\PY{n}{f}\PY{p}{,}\PY{n}{cohr}\PY{p}{)}
         \PY{n}{plt}\PY{o}{.}\PY{n}{xlim}\PY{p}{(}\PY{p}{[}\PY{l+m+mi}{0}\PY{p}{,} \PY{l+m+mi}{100}\PY{p}{]}\PY{p}{)}
         \PY{p}{[}\PY{n}{cohr}\PY{p}{,} \PY{n}{f}\PY{p}{,} \PY{n}{SYY}\PY{p}{,} \PY{n}{SNN}\PY{p}{,} \PY{n}{SYN}\PY{p}{]} \PY{o}{=} \PY{n}{coherence}\PY{p}{(}\PY{n}{n}\PY{p}{,}\PY{n}{y\PYZus{}scaled}\PY{p}{,}\PY{n}{t}\PY{p}{)}      \PY{c+c1}{\PYZsh{} Compute spike\PYZhy{}field cohernece with scaled y.}
         \PY{n}{plot}\PY{p}{(}\PY{n}{f}\PY{p}{,}\PY{n}{cohr}\PY{p}{,}\PY{l+s+s1}{\PYZsq{}}\PY{l+s+s1}{.}\PY{l+s+s1}{\PYZsq{}}\PY{p}{)}\PY{p}{;}
         \PY{n}{xlabel}\PY{p}{(}\PY{l+s+s1}{\PYZsq{}}\PY{l+s+s1}{Frequency [Hz]}\PY{l+s+s1}{\PYZsq{}}\PY{p}{)}
         \PY{n}{ylabel}\PY{p}{(}\PY{l+s+s1}{\PYZsq{}}\PY{l+s+s1}{Coherence}\PY{l+s+s1}{\PYZsq{}}\PY{p}{)}\PY{p}{;}
\end{Verbatim}


    \begin{center}
    \adjustimage{max size={0.9\linewidth}{0.9\paperheight}}{output_56_0.png}
    \end{center}
    { \hspace*{\fill} \\}
    
    We find that this multiplicative change in the amplitude of the field
data does not impact the spike-field coherence. This result is
consistent with our intuition from field-field coherence; the height of
the field does not matter. Instead, it's the consistency of the phase
relation between two signals across trials that is critical for
establishing the coherence.

    Now, let's consider manipulating the spiking data. Right away, we notice
a difference compared to the field data. In this case, a direct
multiplicative change of the spiking data does not make sense. For
example, consider multiplying the spike train data (\texttt{n}) by a
factor of 0.1. Recall that the spike train data consist of two values: 0
or 1. Therefore, the new data after the scaling consist of two values:
\{0, 0.1\} and the interpretation of the variable \texttt{n} no longer
makes sense. What does it mean to have 0.1 spikes in a time interval?

Instead, to scale the spiking data, we change the average firing rate.
We do so in a particular way: by removing spikes from the data, a
process we refer to as \textbf{thinning}. The
\href{https://www.ncbi.nlm.nih.gov/pmc/articles/PMC3800189/}{process of
thinning} is useful when comparing the spike-field coherence computed
for two neurons with different firing rates. A reasonable, intuitive
worry is that the firing rate of a neuron will impact the spike-field
coherence. For example, we might consider that a neuron with a higher
firing rate has the advantage of more opportunities to align with the
field and therefore necessarily will possess a larger spike-field
coherence. By thinning, we reduce the higher firing rate and establish
the two neurons on an equal footing, both with the same opportunity to
align with the field. The objective of the thinning procedure is to
eliminate the contribution of firing rate differences to the spike-field
coherence and allow direct comparison of spike-field coherence results
computed for different neurons.

Let's now thin the spiking data. Here, we implement a simple procedure
by randomly selecting and removing spikes from each trial of the spiking
data. We assume that in selecting spikes at random to remove, we
eliminate both spikes phase-locked to the field and spikes independent
of the field. In this way, neither spikes coupled to the LFP nor spikes
independent of the LFP receive preferential treatment in the thinning
procedure. So, any relations that exist between the spikes and the field
are presumably preserved, and we might expect this thinning procedure,
on its own, to not affect the spike-field coherence. Let's define a
function to implement this thinning procedure in Python:

    \begin{Verbatim}[commandchars=\\\{\}]
{\color{incolor}In [{\color{incolor}41}]:} \PY{k}{def} \PY{n+nf}{thinned\PYZus{}spike\PYZus{}train}\PY{p}{(}\PY{n}{n}\PY{p}{,} \PY{n}{thinning\PYZus{}factor}\PY{p}{)}\PY{p}{:}
             \PY{n}{n\PYZus{}thinned} \PY{o}{=} \PY{n}{np}\PY{o}{.}\PY{n}{copy}\PY{p}{(}\PY{n}{n}\PY{p}{)}
             \PY{k}{for} \PY{n}{k} \PY{o+ow}{in} \PY{n}{np}\PY{o}{.}\PY{n}{arange}\PY{p}{(}\PY{n}{K}\PY{p}{)}\PY{p}{:}                                \PY{c+c1}{\PYZsh{} For each trial,}
                 \PY{n}{spike\PYZus{}times} \PY{o}{=} \PY{n}{np}\PY{o}{.}\PY{n}{where}\PY{p}{(}\PY{n}{n}\PY{p}{[}\PY{n}{k}\PY{p}{,}\PY{p}{:}\PY{p}{]}\PY{o}{==}\PY{l+m+mi}{1}\PY{p}{)}                 \PY{c+c1}{\PYZsh{} ...find the spikes.}
                 \PY{n}{n\PYZus{}spikes} \PY{o}{=} \PY{n}{np}\PY{o}{.}\PY{n}{size}\PY{p}{(}\PY{n}{spike\PYZus{}times}\PY{p}{)}                   \PY{c+c1}{\PYZsh{} ...determine number of spikes.}
                 \PY{n}{spike\PYZus{}times\PYZus{}random} \PY{o}{=} \PY{n}{spike\PYZus{}times}\PY{p}{[}\PY{l+m+mi}{0}\PY{p}{]}\PY{p}{[}\PY{n}{np}\PY{o}{.}\PY{n}{random}\PY{o}{.}\PY{n}{permutation}\PY{p}{(}\PY{n}{n\PYZus{}spikes}\PY{p}{)}\PY{p}{]}    \PY{c+c1}{\PYZsh{} ...permute spikes indices,}
                 \PY{n}{n\PYZus{}remove}\PY{o}{=}\PY{n+nb}{int}\PY{p}{(}\PY{n}{np}\PY{o}{.}\PY{n}{floor}\PY{p}{(}\PY{n}{thinning\PYZus{}factor}\PY{o}{*}\PY{n}{n\PYZus{}spikes}\PY{p}{)}\PY{p}{)}  \PY{c+c1}{\PYZsh{} ... determine number of spikes to remove,}
                 \PY{n}{n\PYZus{}thinned}\PY{p}{[}\PY{n}{k}\PY{p}{,}\PY{n}{spike\PYZus{}times\PYZus{}random}\PY{p}{[}\PY{l+m+mi}{0}\PY{p}{:}\PY{n}{n\PYZus{}remove}\PY{o}{\PYZhy{}}\PY{l+m+mi}{1}\PY{p}{]}\PY{p}{]}\PY{o}{=}\PY{l+m+mi}{0}   \PY{c+c1}{\PYZsh{} remove the spikes.}
             \PY{k}{return} \PY{n}{n\PYZus{}thinned}
\end{Verbatim}


    Note that within the \texttt{for-loop}, we first find the indices
corresponding to spikes in trial \texttt{k}. We then randomly permute
these indices, and select the appropriate proportion of these indices
for removal.

Let's apply this thinning procedure.

    \begin{Verbatim}[commandchars=\\\{\}]
{\color{incolor}In [{\color{incolor}47}]:} \PY{p}{[}\PY{n}{cohr}\PY{p}{,} \PY{n}{f}\PY{p}{,} \PY{n}{SYY}\PY{p}{,} \PY{n}{SNN}\PY{p}{,} \PY{n}{SYN}\PY{p}{]} \PY{o}{=} \PY{n}{coherence}\PY{p}{(}\PY{n}{n}\PY{p}{,}\PY{n}{y}\PY{p}{,}\PY{n}{t}\PY{p}{)}               \PY{c+c1}{\PYZsh{} Plot the coherence for original spike train.}
         \PY{n}{plot}\PY{p}{(}\PY{n}{f}\PY{p}{,}\PY{n}{cohr}\PY{p}{)}
         \PY{p}{[}\PY{n}{cohr}\PY{p}{,} \PY{n}{f}\PY{p}{,} \PY{n}{SYY}\PY{p}{,} \PY{n}{SNN}\PY{p}{,} \PY{n}{SYN}\PY{p}{]} \PY{o}{=} \PY{n}{coherence}\PY{p}{(}\PY{n}{thinned\PYZus{}spike\PYZus{}train}\PY{p}{(}\PY{n}{n}\PY{p}{,}\PY{l+m+mf}{0.9}\PY{p}{)}\PY{p}{,}\PY{n}{y}\PY{p}{,}\PY{n}{t}\PY{p}{)}  \PY{c+c1}{\PYZsh{} ... and for the thinned spike train.}
         \PY{n}{plot}\PY{p}{(}\PY{n}{f}\PY{p}{,}\PY{n}{cohr}\PY{p}{,} \PY{l+s+s1}{\PYZsq{}}\PY{l+s+s1}{r}\PY{l+s+s1}{\PYZsq{}}\PY{p}{)}
         \PY{n}{plt}\PY{o}{.}\PY{n}{xlim}\PY{p}{(}\PY{p}{[}\PY{l+m+mi}{35}\PY{p}{,} \PY{l+m+mi}{55}\PY{p}{]}\PY{p}{)}
         \PY{n}{xlabel}\PY{p}{(}\PY{l+s+s1}{\PYZsq{}}\PY{l+s+s1}{Frequency [Hz]}\PY{l+s+s1}{\PYZsq{}}\PY{p}{)}
         \PY{n}{ylabel}\PY{p}{(}\PY{l+s+s1}{\PYZsq{}}\PY{l+s+s1}{Coherence}\PY{l+s+s1}{\PYZsq{}}\PY{p}{)}\PY{p}{;}
\end{Verbatim}


    \begin{center}
    \adjustimage{max size={0.9\linewidth}{0.9\paperheight}}{output_61_0.png}
    \end{center}
    { \hspace*{\fill} \\}
    
    We plot in the figure above the spike-field coherence for two different
levels of thinning, one of which corresponds to the choice of a thinning
factor of 0.5. We find that, for the thinned spike train, the peak of
spike-field coherence decreases. Why? Intuition suggests that removing
spikes at random (i.e., removing spikes coupled to the phase of LFP and
removing spikes independent of the LFP) should preserve the spike-field
coherence. Perhaps we were unlucky in the thinning procedure and
selected to remove more phase-locked spikes than non-phase-locked
spikes?

    \textbf{Q.} Repeat the analysis with \texttt{thinning\_factor\ =\ 0.5}
to select another random batch of spikes to remove. How does the
spike-field coherence change compared to the original spike train data?
Try this a couple of times, and investigate the peak spike-field
coherence at 45 Hz. Is the peak in the spike-field coherence always
reduced upon thinning?

    Repeating the thinning procedure and selecting new instances of random
spikes to remove preserves the qualitative result. The peak spike-field
coherence at 45 Hz decreases. Perhaps we made a conceptual error in the
thinning procedure or an error in the MATLAB code? In fact, this result
is not a numerical artifact or an error in the code or a problem with
the estimate; it's a property of the spike-field coherence. In
\href{https://www.ncbi.nlm.nih.gov/pubmed/21671792}{Lepage et al, 2011}
it's proven that the spike-field coherence depends on the firing rate.
An important result from
\href{https://www.ncbi.nlm.nih.gov/pubmed/21671792}{Lepage et al, 2011}
is:

    As the firing rate tends to zero, so does the spike-field coherence.

    Therefore, we must be very careful when interpreting the spike-field
coherence, especially when comparing the spike-field coherence of two
neurons with different firing rates. A reduction in spike-field
coherence may occur either through a reduction in association between
the spikes and the field, or through a reduction in the firing rate with
no change in association between the spikes and the field. This is an
important and perhaps counter-intuitive result of spike-field coherence.
The problems at the end of this module further illustrate this result
through simulation. In addition, some procedures exist to mitigate the
dependence of spike-field coherence on the firing rate, as discussed in
the next section.

    Spike-field coherence responds to overall neural spiking activity,
making comparisons between two pairs of spike-field time series
difficult when the average spike-rate differs in the two spike-field
pairs \href{https://www.ncbi.nlm.nih.gov/pubmed/21671792}{Lepage et al,
2011}.

    \subsubsection{Point Process Models of the Spike-Field
Coherence}\label{point-process-models-of-the-spike-field-coherence}

A variety of techniques exist to address the impact of firing rate on
the spike-field coher- ence. We have already outlined the thinning
procedure, a transformation-based technique in which the firing rates of
two neurons are made equal by randomly removing spikes. Here, we focus
on an additional technique that utilizes the generalized linear modeling
framework. We choose this technique (described in detail in
\href{https://www.ncbi.nlm.nih.gov/pmc/articles/PMC3800189/}{Lepage et
al, 2013}) because it allows us to utilize the GLM framework (see
\href{to\%20add}{Module} and \href{to\%20add}{Module}). The fundamental
idea of this procedure is to model the conditional intensity of the
point process as a function of the LFP phase. More specifically, we
consider the model:

\[
\lambda_t = e^{\beta_0 + \beta_1 \cos(\phi(t)) + \beta_2 \sin(\phi(t))} \, ,
\]

where \(\phi(t)\) is the instantaneous phase of a narrowband signal in
the LFP. To compute the phase, we bandpass filter the LFP and apply the
Hilbert transform, as described above in our computation of the
field-triggered average (FTA). Then, using the canonical log link, we
fit the GLM to the spike train data to estimate the model parameters. We
note that the first parameter \(\beta_0\) accounts for the overall
activity of the neuron, while the other two parameters \(\beta_1\) and
\(\beta_2\) capture the association between the LFP phase and spiking
activity. In this way, the overall firing rate and the impact of the
field on the spiking activity are separately modeled, which mitigates
the impact of firing rate on the measure of spike-field association, as
we'll see in the next series of examples.

For the analysis of spike-field association, we select a small frequency
band of interest, bandpass-filter the field data, and then estimate the
phase; the procedures to do so are identical to those used to compute
the FTA. Building from those steps, we now focus on the procedures to
estimate the phase and GLM in Python:

    \begin{Verbatim}[commandchars=\\\{\}]
{\color{incolor}In [{\color{incolor} }]:} \PY{n}{Sxx\PYZus{}model\PYZus{}60Hz} \PY{o}{=} \PY{n}{model}\PY{o}{.}\PY{n}{params}\PY{p}{[}\PY{l+s+s1}{\PYZsq{}}\PY{l+s+s1}{sin}\PY{l+s+s1}{\PYZsq{}}\PY{p}{]} \PY{o}{*}\PY{o}{*} \PY{l+m+mi}{2} \PY{o}{+} \PYZbs{}
                         \PY{n}{model}\PY{o}{.}\PY{n}{params}\PY{p}{[}\PY{l+s+s1}{\PYZsq{}}\PY{l+s+s1}{cos}\PY{l+s+s1}{\PYZsq{}}\PY{p}{]} \PY{o}{*}\PY{o}{*} \PY{l+m+mi}{2}
        \PY{n}{Sxx\PYZus{}model\PYZus{}60Hz}
\end{Verbatim}


    The power estimate from the model consists of two terms: the squared
coefficient of the sine function plus the squared coefficient of the
cosine function. Note that the variable \texttt{Sxx\_model\_60Hz} has
units of \(mV^2\).

    \textbf{Q.} Compare the power estimate from the model (the variable
\texttt{Sxx\_model\_60Hz}) to the power spectral density at 60 Hz
computed using the Fourier transform (code). What do you find?
\textbf{A.} We note that the units of the power spectral density
(variable \texttt{Sxx}) are \(mV^2/Hz\), while the units of the power
estimated in variable \texttt{Sxx\_model\_60Hz} are \(mV^2\). To convert
the power spectral density to (integrated) spectral power, we must
integrate the variable \texttt{Sxx} over a frequency range. Here, we
choose a 1 Hz interval centered at 60 Hz, which corresponds to a single
index of the variable \texttt{faxis}; the frequency resolution for these
data is 0.5 Hz (see next section). Then the approximate integrated power
over this 1 Hz interval is \texttt{Sxx(121)=0.9979}, identical to the
value in Sxx\_model\_60Hz, and with the same units.

    This example, in which we focused on the 60 Hz activity in the EEG,
illustrates how we may use multiple linear regression to estimate the
power. We could extend this procedure to include additional rhythms in
the model beyond 60 Hz (e.g., sine and cosine functions at 1 Hz, 2 Hz, 3
Hz, etc.). In doing so, we would add more terms to the multiple linear
regression model and have more \(\beta\)'s to determine from the data.
Multiple linear regression provides a way to decompose the EEG data into
sine and cosine functions at different frequencies---just as we proposed
to do using the Fourier transform---and then compute the power at each
frequency. Using either multiple linear regression or the Fourier
transform, we aim to decompose the EEG into sine and cosine functions
oscillating at different frequencies.

    \paragraph{Discrete Fourier Transform in Python
}\label{discrete-fourier-transform-in-python}

    Computing the spectrum of a signal \(x\) in Python can be achieved in
two simple steps. The first step is to compute the Fourier transform of
\(x\):

    \begin{Verbatim}[commandchars=\\\{\}]
{\color{incolor}In [{\color{incolor} }]:} \PY{n}{x} \PY{o}{=} \PY{n}{EEG}
        \PY{n}{xf} \PY{o}{=} \PY{n}{np}\PY{o}{.}\PY{n}{real}\PY{p}{(}\PY{n}{np}\PY{o}{.}\PY{n}{fft}\PY{o}{.}\PY{n}{rfft}\PY{p}{(}\PY{n}{x} \PY{o}{\PYZhy{}} \PY{n}{x}\PY{o}{.}\PY{n}{mean}\PY{p}{(}\PY{p}{)}\PY{p}{)}\PY{p}{)}
\end{Verbatim}


    We subtract the mean from \texttt{x} before computing the Fourier
transform. This is not necessary but often useful. For these neural
data, we're not interested in the very slow (0 Hz) activity; instead,
we're interested in rhythmic activity. By subtracting the mean, we
eliminate this low-frequency activity from the subsequent analysis.

The second step is to compute the spectrum, the Fourier transform of
\(x\) multiplied by its complex conjugate:

    \begin{Verbatim}[commandchars=\\\{\}]
{\color{incolor}In [{\color{incolor} }]:} \PY{n}{Sxx} \PY{o}{=} \PY{l+m+mi}{2} \PY{o}{*} \PY{n}{dt} \PY{o}{*}\PY{o}{*} \PY{l+m+mi}{2} \PY{o}{/} \PY{n}{T} \PY{o}{*} \PY{p}{(}\PY{n}{xf} \PY{o}{*} \PY{n}{np}\PY{o}{.}\PY{n}{conj}\PY{p}{(}\PY{n}{xf}\PY{p}{)}\PY{p}{)}
        \PY{n}{plot}\PY{p}{(}\PY{n}{Sxx}\PY{p}{)}
        \PY{n}{xlabel}\PY{p}{(}\PY{l+s+s1}{\PYZsq{}}\PY{l+s+s1}{Indices}\PY{l+s+s1}{\PYZsq{}}\PY{p}{)}
        \PY{n}{ylabel}\PY{p}{(}\PY{l+s+s1}{\PYZsq{}}\PY{l+s+s1}{Power [\PYZdl{}}\PY{l+s+s1}{\PYZbs{}}\PY{l+s+s1}{mu V\PYZca{}2\PYZdl{}/Hz]}\PY{l+s+s1}{\PYZsq{}}\PY{p}{)}
        \PY{n}{show}\PY{p}{(}\PY{p}{)}
\end{Verbatim}


    \textbf{TBD: FIX THIS} \emph{The MATLAB fft function doesn't assume tha
tthe input is real, but Python rfft does this, so there is no need for
all of this exposition, but still some might be useful.}

Upon examining the horizontal axis in this plot, we find it corresponds
to the indices of \texttt{x}, beginning at index 0 and ending at index
\texttt{N\ =\ 1999}. Computing the Fourier transform and multiplying by
the complex conjugate does not change the length of the data \texttt{x}.

Inspection of the plot above reveals a strange characteristic: there are
two large peaks, and the plot exhibits a particular symmetry. If we were
to cut this plot from a printed page and fold the resulting piece of
paper at index 1000, we would find that the two peaks in Sxx align; both
peaks appear to be the same number of indices away from the center index
1000. This suggests that a redundancy occurs in the variable
\texttt{Sxx}. In fact, this redundancy is due to the way the
\texttt{fft} module relates the indices and frequencies of \texttt{Sxx}.
To define this relation requires two new quantities:

\begin{itemize}
\tightlist
\item
  the \emph{frequency resolution}, \(df = \frac{1}{T}\), or the
  reciprocal of the total recording duration;
\item
  the \emph{Nyquist frequency},
  \(f_{NQ} = \frac{f_0}{2} = {1}{2\Delta}\), or half of the sampling
  frequency \(f_0 = \frac{1}{\Delta}\).
\end{itemize}

For the clinical EEG data considered here, the total recording duration
is 2 s (\(T = 2\) s), so the frequency resolution
\(df = 1 / (2\ s) = 0.5\ Hz\). The sampling frequency \(f_0\) is 1000
Hz, so \(f_{NQ} = 1000 / 2\ Hz = 500\ Hz\). There's much more to say
about both quantities, but for now let's simply use both quantities to
consider how Python relates the indices and frequencies of the vector
\texttt{Sxx}.

For the first half of \texttt{Sxx}, the frequency axis increases in
steps of the frequency resolution \(df\) (here, 0.5 Hz) until reaching
the Nyquist frequency \(f_{NQ}\) (here, 500 Hz). This maximal frequency
occurs just past the halfway point of the indices, at index
\(j = N/2 + 1\). Beyond this index, the frequency axis is negative, and
the magnitude of the frequency becomes smaller and smaller until the
value \(-df\) is reached at index \(j = N\).

We may now utilize a useful property of the Fourier transform. When a
signal is real (i.e., the signal has zero imaginary component), the
negative frequencies in the spectrum are redundant. So, the power we
observe at frequency f0 is identical to the power we observe at
frequency f0. For this reason, we can safely ignore the negative
frequencies; these frequencies provide no additional information.
Because the EEG data are real, we conclude that the negative frequencies
in the variable Sxx are redundant and can be ignored. As a specific
example, the value of Sxx at index j = 3 is the same as the value of Sxx
at index j = N 1; these indices correspond to frequencies 2df and 2df ,
respectively (see table 3.1). We therefore need only plot the variable
Sxx for the positive frequencies, more specifically, from index 1 to
index N/2 + 1. This conclusion matches our visual inspection of Sxx in
figure 3.10.

    Given the total duration of the recording (\(T\)) and the sampling
frequency (\(f_0\)) for the data, we can define the frequency axis for
the spectrum \texttt{Sxx}. Now, to compute and plot the spectrum, we
again utilize some code introduced earlier:

    \begin{Verbatim}[commandchars=\\\{\}]
{\color{incolor}In [{\color{incolor} }]:} \PY{n}{xf} \PY{o}{=} \PY{n}{np}\PY{o}{.}\PY{n}{fft}\PY{o}{.}\PY{n}{rfft}\PY{p}{(}\PY{n}{x} \PY{o}{\PYZhy{}} \PY{n}{x}\PY{o}{.}\PY{n}{mean}\PY{p}{(}\PY{p}{)}\PY{p}{)}
        \PY{n}{Sxx} \PY{o}{=} \PY{n}{np}\PY{o}{.}\PY{n}{real}\PY{p}{(}\PY{l+m+mi}{2} \PY{o}{*} \PY{n}{dt} \PY{o}{*}\PY{o}{*} \PY{l+m+mi}{2} \PY{o}{/} \PY{n}{T} \PY{o}{*} \PY{p}{(}\PY{n}{xf} \PY{o}{*} \PY{n}{np}\PY{o}{.}\PY{n}{conj}\PY{p}{(}\PY{n}{xf}\PY{p}{)}\PY{p}{)}\PY{p}{)}
        \PY{n}{df} \PY{o}{=} \PY{l+m+mi}{1} \PY{o}{/} \PY{n}{T}
        \PY{n}{fNQ} \PY{o}{=} \PY{l+m+mi}{1} \PY{o}{/} \PY{n}{dt} \PY{o}{/} \PY{l+m+mi}{2}
        \PY{n}{faxis} \PY{o}{=} \PY{n}{np}\PY{o}{.}\PY{n}{arange}\PY{p}{(}\PY{n+nb}{len}\PY{p}{(}\PY{n}{Sxx}\PY{p}{)}\PY{p}{)} \PY{o}{*} \PY{n}{df}
        \PY{n}{plot}\PY{p}{(}\PY{n}{faxis}\PY{p}{,} \PY{n}{Sxx}\PY{p}{)}
        \PY{n}{xlabel}\PY{p}{(}\PY{l+s+s1}{\PYZsq{}}\PY{l+s+s1}{Frequency (Hz)}\PY{l+s+s1}{\PYZsq{}}\PY{p}{)}
        \PY{n}{ylabel}\PY{p}{(}\PY{l+s+s1}{\PYZsq{}}\PY{l+s+s1}{Power [\PYZdl{}}\PY{l+s+s1}{\PYZbs{}}\PY{l+s+s1}{mu V\PYZca{}2\PYZdl{}/Hz]}\PY{l+s+s1}{\PYZsq{}}\PY{p}{)}
        \PY{n}{show}\PY{p}{(}\PY{p}{)}
\end{Verbatim}


    In the next two section, we focus on interpreting and adjusting the
quantities \(df\) and \(f_{NQ}\). Doing so is critical to develop
further an intuition for the spectrum.

    \paragraph{\texorpdfstring{The Nyquist frequency, \(f_{NQ}\)
}{The Nyquist frequency, f\_\{NQ\} }}\label{the-nyquist-frequency-f_nq}

    The formula for the Nyquist frequency is

\[f_{NQ} = \frac{f_0}{2}.\]

The Nyquist frequency is the highest frequency we can possibly hope to
observe in the data. To illustrate this, let's consider a true EEG
signal that consists of a very simple time series---a pure sinusoid that
oscillates at some frequency \(f_s\). Of course, we never observe the
true signal. Instead, we observe a sampling of this signal, which
depends on the sampling interval \(\Delta\). We consider three cases for
different values of \(\Delta\). In the first case, we purchase a very
expensive piece of equipment that can sample the true signal at a high
rate, \(f_0 \gg f_s\). In this case, we cover the true brain signal with
many samples and given these samples, we can accurately reconstruct the
underlying data.

    \begin{Verbatim}[commandchars=\\\{\}]
{\color{incolor}In [{\color{incolor} }]:} \PY{n}{tempfs} \PY{o}{=} \PY{l+m+mi}{6}
        \PY{n}{pi} \PY{o}{=} \PY{n}{np}\PY{o}{.}\PY{n}{pi}
        \PY{n}{tempsmooth} \PY{o}{=} \PY{n}{np}\PY{o}{.}\PY{n}{linspace}\PY{p}{(}\PY{l+m+mi}{0}\PY{p}{,}\PY{l+m+mi}{1}\PY{p}{,}\PY{l+m+mi}{1000}\PY{p}{)}
        \PY{n}{tempx} \PY{o}{=} \PY{n}{np}\PY{o}{.}\PY{n}{linspace}\PY{p}{(}\PY{l+m+mi}{0}\PY{p}{,} \PY{l+m+mi}{1}\PY{p}{,} \PY{l+m+mi}{8}\PY{o}{*}\PY{n}{tempfs} \PY{o}{+} \PY{l+m+mi}{1}\PY{p}{)}
        \PY{n}{tempyx} \PY{o}{=} \PY{n}{np}\PY{o}{.}\PY{n}{cos}\PY{p}{(}\PY{l+m+mi}{2} \PY{o}{*} \PY{n}{pi} \PY{o}{*} \PY{n}{tempfs} \PY{o}{*} \PY{n}{tempx}\PY{p}{)}
        \PY{n}{tempysm} \PY{o}{=} \PY{n}{np}\PY{o}{.}\PY{n}{cos}\PY{p}{(}\PY{l+m+mi}{2} \PY{o}{*} \PY{n}{pi} \PY{o}{*} \PY{n}{tempfs} \PY{o}{*} \PY{n}{tempsmooth}\PY{p}{)}
        
        \PY{n}{plot}\PY{p}{(}\PY{n}{tempsmooth}\PY{p}{,} \PY{n}{tempysm}\PY{p}{,} \PY{l+s+s1}{\PYZsq{}}\PY{l+s+s1}{k}\PY{l+s+s1}{\PYZsq{}}\PY{p}{,} \PY{n}{lw}\PY{o}{=}\PY{l+m+mi}{3}\PY{p}{)}
        \PY{n}{plot}\PY{p}{(}\PY{n}{tempx}\PY{p}{,} \PY{n}{tempyx}\PY{p}{,} \PY{l+s+s1}{\PYZsq{}}\PY{l+s+s1}{go\PYZhy{}}\PY{l+s+s1}{\PYZsq{}}\PY{p}{)}
        \PY{n}{show}\PY{p}{(}\PY{p}{)}
\end{Verbatim}


    \paragraph{The frequency resolution}\label{the-frequency-resolution}

    \subsubsection{Step 5: Decibel scaling}\label{step-5-decibel-scaling}

    \subsubsection{Step 6: The spectrogram}\label{step-6-the-spectrogram}

    Details and intuitions behind each step are proved in the supplement
entitled
\href{Supplement.\%20Intuition\%20behind\%20the\%20power\%20spectral\%20density.ipynb}{\emph{Intuition
behind the power spectral density}}.


    % Add a bibliography block to the postdoc
    
    
    
    \end{document}
